\begin{tabular}{llr}
\toprule
 &  & payment \\
year & programName &  \\
\midrule
2008 & DIRECT AND COUNTER CYCLICAL PROG & \$ 0.00 M \\
\cline{1-3}
\multirow[t]{2}{*}{2009} & DCP PROGRAM - DIRECT PAYMENTS & \$ 0.00 M \\
 & DIRECT AND COUNTER CYCLICAL PROG & \$ 0.00 M \\
\cline{1-3}
2010 & DCP PROGRAM - DIRECT PAYMENTS & \$ 0.00 M \\
\cline{1-3}
\multirow[t]{2}{*}{2011} & TRADE ADJUSTMENT ASSISTANCE FOR FARMERS & \$ 0.00 M \\
 & DCP - DIRECT & \$ 0.00 M \\
\cline{1-3}
2012 & TRADE ADJUSTMENT ASSISTANCE FOR FARMERS & \$ 0.01 M \\
\cline{1-3}
2013 & TRADE ADJUSTMENT ASSISTANCE FOR FARMERS & \$ 0.00 M \\
\cline{1-3}
2014 & TRADE ADJUSTMENT ASSISTANCE FOR FARMERS & \$ 0.01 M \\
\cline{1-3}
\multirow[t]{2}{*}{2016} & EMERG ASSIST LIVESTOCK BEES FISH (ELAP) & \$ 0.00 M \\
 & TRADE ADJUSTMENT ASSISTANCE FOR FARMERS & \$ 0.00 M \\
\cline{1-3}
2019 & EMERG ASSIST LIVESTOCK BEES FISH (ELAP) & \$ 0.00 M \\
\cline{1-3}
\multirow[t]{3}{*}{2020} & CFAPCCA2 & \$ 0.51 M \\
 & SEAFOOD TRADE RELIEF PROGRAM & \$ 0.03 M \\
 & CFAPCARES & \$ 0.01 M \\
\cline{1-3}
\multirow[t]{3}{*}{2021} & CFAPCCA2 & \$ 0.02 M \\
 & SEAFOOD TRADE RELIEF PROGRAM & \$ 0.00 M \\
 & LIVESTOCK FORAGE PROGRAM & \$ 0.00 M \\
\cline{1-3}
\bottomrule
\end{tabular}
