\begin{tabular}{llr}
\toprule
 &  & payment \\
year & programName &  \\
\midrule
2005 & LIVESTOCK ASSISTANCE PROGRAM & \$ 0.68 M \\
\cline{1-3}
\multirow[t]{3}{*}{2008} & CONSERVATION RESERVE PROGRAM - ANNUAL LAND RENTAL & \$ 0.36 M \\
 & 05 - 07 LIVESTOCK COMPENSATION PROGRAM & \$ 0.08 M \\
 & DIRECT AND COUNTER CYCLICAL PROG & \$ 0.06 M \\
\cline{1-3}
\multirow[t]{3}{*}{2009} & LIVESTOCK FORAGE DISASTER  PROGRAM & \$ 0.42 M \\
 & CRP PAYMENT - ANNUAL PAYMENT & \$ 0.35 M \\
 & DCP PROGRAM - DIRECT PAYMENTS & \$ 0.06 M \\
\cline{1-3}
\multirow[t]{3}{*}{2010} & CRP PAYMENT - ANNUAL RENTAL & \$ 0.36 M \\
 & LIVESTOCK FORAGE DISASTER PROGRAM & \$ 0.33 M \\
 & NONINSURED ASSISTANCE PROGRAM & \$ 0.26 M \\
\cline{1-3}
\multirow[t]{3}{*}{2011} & LIVESTOCK FORAGE DISASTER PROGRAM & \$ 1.44 M \\
 & CRP PAYMENT - ANNUAL RENTAL & \$ 0.32 M \\
 & EMERG ASSIST LIVESTOCK BEES FISH (ELAP) & \$ 0.24 M \\
\cline{1-3}
\multirow[t]{3}{*}{2012} & NON-INSURED ASSISTANCE PROGRAM & \$ 0.76 M \\
 & CRP PAYMENT - ANNUAL RENTAL & \$ 0.34 M \\
 & LIVESTOCK FORAGE DISASTER PROGRAM & \$ 0.09 M \\
\cline{1-3}
\multirow[t]{3}{*}{2013} & NON-INSURED ASSISTANCE PROGRAM & \$ 0.87 M \\
 & CRP PAYMENT - ANNUAL RENTAL & \$ 0.35 M \\
 & GRASSLANDS RESERVE PROGRAM & \$ 0.07 M \\
\cline{1-3}
\multirow[t]{3}{*}{2014} & LIVESTOCK FORAGE PROGRAM & \$ 6.27 M \\
 & CRP PAYMENT - ANNUAL RENTAL & \$ 0.36 M \\
 & GRASSLANDS RESERVE PROGRAM & \$ 0.07 M \\
\cline{1-3}
\multirow[t]{3}{*}{2015} & LIVESTOCK FORAGE PROGRAM & \$ 0.95 M \\
 & CRP PAYMENT - ANNUAL RENTAL & \$ 0.36 M \\
 & GRASSLANDS RESERVE PROGRAM & \$ 0.07 M \\
\cline{1-3}
\multirow[t]{3}{*}{2016} & CRP PAYMENT - ANNUAL RENTAL & \$ 0.34 M \\
 & LIVESTOCK FORAGE PROGRAM & \$ 0.13 M \\
 & NON-INSURED ASSISTANCE PROG AUTHORIZED & \$ 0.08 M \\
\cline{1-3}
\multirow[t]{3}{*}{2017} & CRP PAYMENT - ANNUAL RENTAL & \$ 0.28 M \\
 & PRICE LOSS COVERAGE & \$ 0.15 M \\
 & GRASSLANDS RESERVE PROGRAM & \$ 0.07 M \\
\cline{1-3}
\multirow[t]{3}{*}{2018} & LIVESTOCK FORAGE PROGRAM & \$ 0.82 M \\
 & CRP PAYMENT - ANNUAL RENTAL & \$ 0.36 M \\
 & PRICE LOSS COVERAGE & \$ 0.09 M \\
\cline{1-3}
\multirow[t]{3}{*}{2019} & CRP PAYMENT - ANNUAL RENTAL & \$ 0.35 M \\
 & GRASSLANDS RESERVE PROGRAM & \$ 0.07 M \\
 & WEB-BASED NAP & \$ 0.07 M \\
\cline{1-3}
\multirow[t]{3}{*}{2020} & CFAPCCA2 & \$ 1.58 M \\
 & CFAPCCCCA & \$ 1.31 M \\
 & NON-INSURED ASSISTANCE PROGRAM & \$ 0.99 M \\
\cline{1-3}
\multirow[t]{3}{*}{2021} & WEB-BASED NAP & \$ 1.00 M \\
 & LIVESTOCK FORAGE PROGRAM & \$ 0.96 M \\
 & CFAP3 - LTU & \$ 0.63 M \\
\cline{1-3}
\multirow[t]{3}{*}{2022} & LIVESTOCK FORAGE PROGRAM & \$ 2.69 M \\
 & EMERGENCY LIVESTOCK RELIEF PROGRAM & \$ 1.09 M \\
 & CRP PAYMENT - ANNUAL RENTAL & \$ 0.68 M \\
\cline{1-3}
\bottomrule
\end{tabular}
