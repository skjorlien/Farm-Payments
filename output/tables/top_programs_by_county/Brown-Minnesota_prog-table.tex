\begin{tabular}{llr}
\toprule
 &  & payment \\
year & programName &  \\
\midrule
2005 & EWE LAMB REPLACEMENT OR RETENTION & \$ 0.01 M \\
\cline{1-3}
2006 & INCOME LOSS - MILK, PART 2 & \$ 0.61 M \\
\cline{1-3}
2007 & INCOME LOSS - MILK, PART 2 & \$ 0.12 M \\
\cline{1-3}
\multirow[t]{3}{*}{2008} & DIRECT AND COUNTER CYCLICAL PROG & \$ 5.42 M \\
 & CONSERVATION RESERVE PROGRAM - ANNUAL LAND RENTAL & \$ 1.68 M \\
 & 05 - 07 CROP DISASTER ASSISTANCE & \$ 0.58 M \\
\cline{1-3}
\multirow[t]{3}{*}{2009} & DCP PROGRAM - DIRECT PAYMENTS & \$ 4.89 M \\
 & CRP PAYMENT - ANNUAL PAYMENT & \$ 1.78 M \\
 & MILK INCOME LOSS CONTRACT II & \$ 1.38 M \\
\cline{1-3}
\multirow[t]{3}{*}{2010} & DCP - DIRECT & \$ 4.95 M \\
 & CRP PAYMENT - ANNUAL RENTAL & \$ 1.83 M \\
 & AVERAGE CROP REVENUE ELECTION-DIRECT & \$ 0.31 M \\
\cline{1-3}
\multirow[t]{3}{*}{2011} & DCP - DIRECT & \$ 4.95 M \\
 & CRP PAYMENT - ANNUAL RENTAL & \$ 1.87 M \\
 & AVERAGE CROP REVENUE ELECTION-DIRECT & \$ 0.30 M \\
\cline{1-3}
\multirow[t]{3}{*}{2012} & DCP - DIRECT & \$ 5.04 M \\
 & CRP PAYMENT - ANNUAL RENTAL & \$ 1.89 M \\
 & INCOME LOSS - MILK, PART 2 & \$ 0.81 M \\
\cline{1-3}
\multirow[t]{3}{*}{2013} & DCP - DIRECT & \$ 5.08 M \\
 & CRP PAYMENT - ANNUAL RENTAL & \$ 1.95 M \\
 & SUPPLEMENTAL REVENUE ASSISTANCE PROGRAM & \$ 0.44 M \\
\cline{1-3}
\multirow[t]{3}{*}{2014} & CRP PAYMENT - ANNUAL RENTAL & \$ 2.06 M \\
 & AVERAGE CROP REVENUE ELECTION-ACRE & \$ 0.20 M \\
 & LIVESTOCK FORAGE PROGRAM & \$ 0.13 M \\
\cline{1-3}
\multirow[t]{3}{*}{2015} & AGRICULTURAL RISK COVERAGE PROG - COUNTY & \$ 12.21 M \\
 & CRP PAYMENT - ANNUAL RENTAL & \$ 2.02 M \\
 & LIVESTOCK FORAGE PROGRAM & \$ 0.01 M \\
\cline{1-3}
\multirow[t]{3}{*}{2016} & AGRICULTURAL RISK COVERAGE PROG - COUNTY & \$ 7.98 M \\
 & CRP PAYMENT - ANNUAL RENTAL & \$ 2.29 M \\
 & MARGIN PROTECTION PROGRAM - DAIRY & \$ 0.06 M \\
\cline{1-3}
\multirow[t]{3}{*}{2017} & AGRICULTURAL RISK COVERAGE PROG - COUNTY & \$ 4.01 M \\
 & CRP PAYMENT - ANNUAL RENTAL & \$ 2.19 M \\
 & PRICE LOSS COVERAGE & \$ 0.05 M \\
\cline{1-3}
\multirow[t]{3}{*}{2018} & MARKET FACILITATION PROGRAM - CROPS & \$ 9.15 M \\
 & CRP PAYMENT - ANNUAL RENTAL & \$ 2.05 M \\
 & MARKET FACILITATION PROGRAM - DAHG & \$ 1.17 M \\
\cline{1-3}
\multirow[t]{3}{*}{2019} & TMP/MFP 2019 NON SPECIALTY CROPS & \$ 14.61 M \\
 & ARC PROGRAM-COUNTY COVERAGE & \$ 3.32 M \\
 & MARKET FACILITATION PROGRAM - CROPS & \$ 2.86 M \\
\cline{1-3}
\multirow[t]{3}{*}{2020} & CFAPCCA2 & \$ 16.05 M \\
 & CFAPCCCCA & \$ 6.56 M \\
 & CFAPCARES & \$ 6.36 M \\
\cline{1-3}
\multirow[t]{3}{*}{2021} & WHIP PLUS 3 ASSISTANCE & \$ 7.40 M \\
 & CFAP3 - TUP & \$ 6.14 M \\
 & CRP PAYMENT - ANNUAL RENTAL & \$ 2.87 M \\
\cline{1-3}
\multirow[t]{3}{*}{2022} & EMGNCY RELIEF PRGM-NONSPECIALITY CROPS & \$ 4.22 M \\
 & CRP PAYMENT - ANNUAL RENTAL & \$ 3.08 M \\
 & SPOT MARKET HOG PANDEMIC PROGRAM & \$ 1.14 M \\
\cline{1-3}
\bottomrule
\end{tabular}
