\documentclass{article}
\usepackage[utf8]{inputenc}
\usepackage{enumitem, amsfonts, amssymb, float, amsmath, graphicx, multicol, multirow, booktabs}
\usepackage[margin=1in]{geometry}

\title{Project Proposal to Request ARMS Access}
% \author{Scott Kjorlien}
% \date{\today}

\begin{document}
\maketitle

\section*{Data}
Agricultural Resource Management Survey (ARMS) Phase II
Agricultural Resource Management Survey (ARMS) Phase III (I think we just need phase 3) 
phase 3 is what we currently have. phase 2 is about farming practices. 

\section*{Research Team} 
\begin{itemize}
    \item \textbf{Principle Investigator:} Professor Ethan Ligon, University of California, Berkeley. email: ligon@berkeley.edu. phone: XXX-XXXX. citizenship: U.S. data access: yes
    \item \textbf{Graduate Student Researcher:} Scott Kjorlien, University of California, Berkeley. email: scottkjorlien@berkeley.edu. phone: (925) 324-5224. citizenship: U.S. data access: yes 
\end{itemize}

full legal name, affiliation, title, email, phone, citizen ship, currently has active special sworn status. if the research must access. 
for scott, data access = Yes, special sworn status = yes


\section*{Research Description} 
\subsection*{Main Objective}
The principal objective of the project is to measure the value of crop insurance
and other conditional agricultural income support to agricultural producers in the
US. A secondary (and supporting) objective involves the construction of a “quasi panel”
dataset of farm-households, with longitudinal data on farm income and farm household
expenditures.

\subsection*{Recent Federal Programs and Literature}
A rapidly expanding set of federal programs transfers resources to US farmers.
Crop insurance is supposed to compensate farmers for production shortfalls or drops
in prices; the current administration's Market Facilitation Program (MFP) is meant
to compensate farmers for losses due to recently imposed restrictions on trade; the
“Coronavirus Food Assistance Program” (CFAP) and other legislation currently proposed
is supposed to support farmers in the face of losses caused by Covid-19 and its
disruptions to the supply chain.
The scale of recent support and proposed transfers is huge. “Baseline” federal
support for agriculture in 2018 was roughly \$8.75 billion. MFP was introduced only
in late 2018; expenditures under this program have been roughly \$14.5 billion to
date. The just introduced CFAP program is a roughly \$19 billion program, of which
about \$16 billion is to be given directly to producers. Additional legislation to deal
with the fall-out of the Covid-19 crisis is likely to lead to further transfers. These
programs will more than double federal support to US agriculture in 2019-20.
Crop insurance and other federal supported payments in the US are supposed to
reduce the risk faced by farmers, and so improve their welfare. There are compelling
theoretical reasons to think that this should be so, but there have been no real
attempts to actually measure the reduction in risk or changes in welfare. This means
that one can't directly evaluate how these programs are performing relative to their
stated aims. And without some measurement of salient outcomes, one may not be
able to identify the strongest-performing programs, detect poorly designed programs
or insurance policies, or to devise improvements.
This project seeks to measure risk and welfare impacts of crop insurance and other
conditional agricultural income support in the US. A recent paper (Carter, Dong,
and Steinbach 2020) finds that the MFP did not fully compensate many California
farmers for losses attributable to the 2018 Trade War with China. Additional work
by Carter seeks to understand which types of US farmers benefited most from trade
aid cash transfers which were created in response to retaliatory tariffs imposed during
the 2018 Trade War. Our project will contribute to this literature by measuring the
value of crop insurance and other conditional agricultural income support programs
to agricultural producers in the US. We will answer questions about which transfers
a producer of a given type can receive, what proportion of transfers go to producers
at different points in the income or expenditure distribution; etc. In classifying types
of farmers, we will only use data from regions and years where enough farms are
observed in the ARMS data to prevent the compromise of confidentiality.
Furthermore, while the direct costs of providing subsidized crop insurance are well
understood, and the impact of these programs on production have been estimated for
California producers, no one has constructed estimates of the value of risk reduction
from crop insurance, either for the US or for California. The rationale for providing
crop insurance and other conditional support rests on the view that there''s value in
reducing the economic risk faced by farmers, and that this risk depends on variation
in yields, prices, or revenue. And so measuring the value of the programs rests on
measuring both the risk farmers bear as well as the reduction in risk delivered by
crop insurance.

\section*{Project Title} 
Measuring the Value of Crop Insurance and Other Conditional Agricultural Income Support to US Farmers

\section*{Project Duration} 
36 months

\section*{Funding Sources} 


\section*{Timeline} 
table with column for each year. row for each item 

\begin{table}
    \begin{tabular}{c|c|c|c}
        \toprule 
        Activity                                &   Year 1  &   Year 2  &   Year 3  \\
        \midrule 
        Data cleaning and descriptive analysis  &   X       &           &           \\ 


    \end{tabular}
\end{table}


\section*{Research Question(s)} 
What is the value of crop insurance and other agricultural income support programs to agricultural households in the U.S.? 

\section*{Demonstrated Need}
In order to assess the value of programs to agricultural households, one needs to construct a dataset of income, expenditures, and indemnity 
payments for individuals and households. ARMS is the best option for this objective.  

\section*{Study Population}
The population of interest for this study is all U.S. farm households. ARMS consists of annual survey responses from which we can construct a ``quasi-panel'' of farmers 
growing a given crop. 


\section*{Project Abstract}


\section*{Time, Geographic, and Other Units Required} 
We request the annual survey data from 2006 through the most recent year. In addition, in order to construct a representative farm household at the smallest possible geographic unit, 
we request zip code-level data. 

\section*{Work locations}
SDK: Home 
EL: 

\section*{Data Linkages}
N/A

\section*{User-Provided Data} 


\section*{Software Requirement} 
We require no special software aside from those already provided in the USDA Virtual Data Enclave, however, we do request 
a current version of Python. 


\section*{Variables Requested}
N/A (only for Census of Ag data)

\section*{Methodology} 



\section*{List of References}


\section*{Project Products} 



\section*{Requested Output} 


\section*{Agency Benefits} 

\section*{Documentation}



\end{document}