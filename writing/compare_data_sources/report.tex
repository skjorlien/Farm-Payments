\documentclass{article}
\usepackage[utf8]{inputenc}
\usepackage{enumitem, amsfonts, tikz, amssymb, hyperref, longtable, float, amsmath, graphicx, multicol, multirow, booktabs}
\usepackage[margin=1in]{geometry}
\usetikzlibrary{quotes}
\usetikzlibrary{positioning}

\title{Data Source Summary Tables}

\author{Scott Kjorlien}
\date{\today}

\begin{document}
\maketitle

\section*{FOIA Data Summary}
\begin{table}[H]
\caption{Summary Stats from FOIA data}
\begin{tabular}{lrrrrr}
\toprule
 & \multicolumn{5}{r}{payment} \\
 & mean & max & sum & std & count \\
year &  &  &  &  &  \\
\midrule
2006 & \$ 788.06 & \$ 91,720.16 & \$ 441,395,789.32 & \$ 1,676.05 & 560102 \\
2007 & \$ 352.16 & \$ 25,273.56 & \$ 74,762,121.50 & \$ 864.11 & 212296 \\
2008 & \$ 867.34 & \$ 668,810.00 & \$ 10,207,878,350.00 & \$ 2,967.92 & 11769158 \\
2009 & \$ 2,022.60 & \$ 697,812.00 & \$ 9,992,060,725.00 & \$ 4,625.21 & 4940213 \\
2010 & \$ 1,606.34 & \$ 1,086,026.00 & \$ 10,615,437,806.00 & \$ 5,096.92 & 6608477 \\
2011 & \$ 1,470.96 & \$ 1,911,038.00 & \$ 7,853,899,680.98 & \$ 4,967.86 & 5339306 \\
2012 & \$ 2,178.24 & \$ 2,266,952.00 & \$ 8,055,749,016.28 & \$ 6,030.28 & 3698290 \\
2013 & \$ 2,271.05 & \$ 2,170,375.00 & \$ 7,470,720,100.00 & \$ 7,338.33 & 3289544 \\
2014 & \$ 4,208.69 & \$ 5,575,000.00 & \$ 7,179,329,103.00 & \$ 12,448.26 & 1705836 \\
2015 & \$ 3,363.29 & \$ 4,770,000.00 & \$ 8,333,282,360.00 & \$ 8,294.39 & 2477720 \\
2016 & \$ 3,071.41 & \$ 2,255,666.00 & \$ 10,711,725,040.00 & \$ 7,262.81 & 3487563 \\
2017 & \$ 2,968.00 & \$ 2,072,055.00 & \$ 9,527,872,811.00 & \$ 7,634.21 & 3210198 \\
2018 & \$ 3,400.24 & \$ 952,719.00 & \$ 11,600,600,872.59 & \$ 9,075.45 & 3411698 \\
2019 & \$ 4,878.41 & \$ 1,500,000.00 & \$ 19,979,521,019.04 & \$ 14,411.92 & 4095496 \\
2020 & \$ 4,892.06 & \$ 3,141,886.78 & \$ 37,812,702,107.61 & \$ 18,662.83 & 7729400 \\
2021 & \$ 9,509.09 & \$ 1,495,778.52 & \$ 762,857,586.43 & \$ 30,556.64 & 80224 \\
\bottomrule
\end{tabular}
\end{table}


\newpage
\section*{Public Data Summary}
\begin{table}[H]
\caption{Summary Stats from Public data}
\begin{tabular}{lrrrrr}
\toprule
 & \multicolumn{5}{r}{payment} \\
 & mean & max & sum & std & count \\
year &  &  &  &  &  \\
\midrule
2004 & \$ 1,993.27 & \$ 19,020.68 & \$ 211,287.05 & \$ 3,312.42 & 106 \\
2005 & \$ 1,858.11 & \$ 347,834.20 & \$ 1,239,418,612.72 & \$ 4,715.05 & 667033 \\
2006 & \$ 788.06 & \$ 91,720.16 & \$ 441,395,789.32 & \$ 1,676.05 & 560102 \\
2007 & \$ 352.16 & \$ 25,273.56 & \$ 74,762,121.50 & \$ 864.11 & 212296 \\
2008 & \$ 867.34 & \$ 668,810.00 & \$ 10,207,879,211.09 & \$ 2,967.92 & 11769158 \\
2009 & \$ 2,022.60 & \$ 697,812.00 & \$ 9,992,061,451.50 & \$ 4,625.21 & 4940213 \\
2010 & \$ 1,654.83 & \$ 4,925,876.00 & \$ 10,965,568,866.50 & \$ 7,522.88 & 6626408 \\
2011 & \$ 1,560.55 & \$ 5,000,000.00 & \$ 8,855,627,288.19 & \$ 7,734.06 & 5674683 \\
2012 & \$ 2,223.25 & \$ 5,000,000.00 & \$ 8,770,993,864.65 & \$ 9,345.58 & 3945131 \\
2013 & \$ 2,305.05 & \$ 4,000,000.00 & \$ 9,017,546,913.82 & \$ 9,509.81 & 3912079 \\
2014 & \$ 4,340.19 & \$ 5,575,000.00 & \$ 7,806,551,895.88 & \$ 15,419.75 & 1798664 \\
2015 & \$ 3,393.23 & \$ 4,770,000.00 & \$ 8,989,304,069.96 & \$ 10,954.09 & 2649186 \\
2016 & \$ 3,101.87 & \$ 6,000,000.00 & \$ 11,508,682,520.04 & \$ 12,510.59 & 3710243 \\
2017 & \$ 3,012.22 & \$ 3,955,922.00 & \$ 10,263,637,351.99 & \$ 10,930.82 & 3407337 \\
2018 & \$ 3,537.25 & \$ 94,000,000.00 & \$ 12,751,250,456.50 & \$ 92,089.58 & 3604845 \\
2019 & \$ 4,848.17 & \$ 3,500,000.00 & \$ 20,896,707,040.36 & \$ 15,848.31 & 4310226 \\
2020 & \$ 4,980.09 & \$ 82,324,957.00 & \$ 39,357,702,923.99 & \$ 80,209.21 & 7903003 \\
2021 & \$ 4,069.86 & \$ 9,924,038.00 & \$ 16,422,246,507.68 & \$ 14,343.61 & 4035087 \\
2022 & \$ 7,240.45 & \$ 4,243,602.50 & \$ 13,187,642,193.08 & \$ 22,089.06 & 1821385 \\
\bottomrule
\end{tabular}
\end{table}


\section*{Notes \& Takeaway}
Beginning in 2010, the two different data sources diverge. The Public Data, found at 
\href{https://www.fsa.usda.gov/news-room/efoia/electronic-reading-room/frequently-requested-information/payment-files-information/index}{this site}
has more records, a greater sum, larger mean payments, and often a higher maximum than does the 
FOIA'd data. As of July 31, 2023, I have not investigated the source of the discrepancies. My initial hunch  
is that one or more programs were omitted in the FOIA. 

Without digging deeper into the discrepancies, I recommend using the Public Data because 
\begin{itemize}
    \item We can use more recent data 
    \item The public data seems more complete 
    \item It will be easier to incorporate future data sources if USDA updates the public webpage
    \item There are no additional useful fields from the FOIA'd data that we need to make use of
\end{itemize}

\end{document}