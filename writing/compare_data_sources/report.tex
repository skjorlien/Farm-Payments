\documentclass{article}
\usepackage[utf8]{inputenc}
\usepackage{enumitem, amsfonts, tikz, amssymb, hyperref, longtable, float, amsmath, graphicx, multicol, multirow, booktabs}
\usepackage[margin=1in]{geometry}
\usetikzlibrary{quotes}
\usetikzlibrary{positioning}

\title{Data Source Summary Tables}

\author{Scott Kjorlien}
\date{\today}

\begin{document}
\maketitle

\section*{TEST}
\begin{longtable}{lrrrrrrrrrrrrrrrrrrr}
\caption{Public minus FOIA in Millions of Dollars by State} \label{stateYearDiffTable} \\
\toprule
year & 2004 & 2005 & 2006 & 2007 & 2008 & 2009 & 2010 & 2011 & 2012 & 2013 & 2014 & 2015 & 2016 & 2017 & 2018 & 2019 & 2020 & 2021 & 2022 \\
stateabbr &  &  &  &  &  &  &  &  &  &  &  &  &  &  &  &  &  &  &  \\
\midrule
\endfirsthead
\caption[]{Public minus FOIA in Millions of Dollars by State} \\
\toprule
year & 2004 & 2005 & 2006 & 2007 & 2008 & 2009 & 2010 & 2011 & 2012 & 2013 & 2014 & 2015 & 2016 & 2017 & 2018 & 2019 & 2020 & 2021 & 2022 \\
stateabbr &  &  &  &  &  &  &  &  &  &  &  &  &  &  &  &  &  &  &  \\
\midrule
\endhead
\midrule
\multicolumn{20}{r}{Continued on next page} \\
\midrule
\endfoot
\bottomrule
\endlastfoot
AK & 0 & 2 & 0 & 0 & -0 & 0 & -0 & 0 & 0 & 0 & 0 & 0 & 0 & 0 & 0 & 0 & 1 & 38 & 2 \\
AL & 0 & 1 & 0 & 0 & 0 & -0 & 13 & 16 & 4 & 9 & 3 & 2 & 4 & 7 & 4 & 6 & 9 & 181 & 83 \\
AR & 0 & 10 & 0 & 0 & 0 & -0 & 7 & 12 & 18 & 39 & 6 & 11 & 14 & 15 & 16 & 25 & 29 & 469 & 308 \\
AS & 0 & 0 & 0 & 0 & 0 & 0 & 0 & 0 & 0 & 0 & 0 & 0 & 0 & 0 & 0 & 0 & 2 & 1 & 0 \\
AZ & 0 & 5 & 0 & 0 & -0 & -0 & -2 & 8 & 6 & 8 & 4 & 3 & 1 & 1 & 1 & 3 & 14 & 72 & 49 \\
CA & 0 & 11 & 0 & 0 & 0 & -0 & 7 & 42 & 20 & 26 & 16 & 12 & 9 & 5 & 9 & 14 & 85 & 562 & 541 \\
CT & 0 & 0 & 0 & 0 & 0 & -0 & 0 & 0 & 0 & 0 & 0 & 0 & 0 & 0 & 0 & 0 & 0 & 7 & 13 \\
DC & 0 & 0 & 0 & 0 & 0 & 0 & 265 & 250 & 258 & 226 & 234 & 232 & 323 & 297 & 690 & 275 & 893 & 159 & 0 \\
DE & 0 & 0 & 0 & 0 & -0 & 0 & 1 & 1 & 2 & 2 & 0 & 0 & 0 & 0 & 0 & 1 & 0 & 18 & 4 \\
FL & 0 & 17 & 0 & 0 & 0 & 0 & 4 & 14 & 5 & 5 & 4 & 3 & 2 & 4 & 7 & 16 & 26 & 153 & 148 \\
GA & 0 & 70 & 0 & 0 & 0 & 0 & -10 & 24 & 7 & 22 & 5 & 6 & 10 & 10 & 8 & 14 & 24 & 393 & 278 \\
HI & 0 & 2 & 0 & 0 & 0 & -0 & -0 & 10 & 0 & 1 & 0 & 0 & 0 & 0 & 0 & 0 & 2 & 12 & 15 \\
IA & 0 & 3 & 0 & 0 & 0 & 0 & 11 & 27 & 28 & 97 & 15 & 36 & 34 & 23 & 35 & 60 & 93 & 1,114 & 888 \\
ID & 0 & 3 & 0 & 0 & 0 & -0 & 30 & 25 & 24 & 35 & 7 & 19 & 32 & 30 & 28 & 26 & 70 & 234 & 183 \\
IL & 0 & 0 & 0 & 0 & 0 & 0 & 12 & 22 & 24 & 84 & 8 & 13 & 30 & 16 & 26 & 46 & 61 & 822 & 370 \\
IN & 0 & 9 & 0 & 0 & 0 & 0 & -8 & 13 & 13 & 40 & 3 & 5 & 13 & 8 & 10 & 19 & 28 & 409 & 129 \\
KS & 0 & 6 & 0 & 0 & 0 & 0 & 20 & 24 & 21 & 89 & 19 & 18 & 17 & 19 & 23 & 42 & 55 & 786 & 527 \\
KY & 0 & 247 & 0 & 0 & 0 & 0 & -39 & 13 & 6 & 18 & 4 & 5 & 5 & 5 & 4 & 7 & 12 & 209 & 101 \\
LA & 0 & 1 & 0 & 0 & 0 & 0 & 2 & 10 & 7 & 18 & 3 & 3 & 5 & 5 & 5 & 5 & 8 & 243 & 202 \\
MA & 0 & 0 & 0 & 0 & -0 & 0 & -0 & 0 & 0 & 1 & 0 & 0 & 0 & 0 & 0 & 0 & 2 & 11 & 14 \\
MD & 0 & 0 & 0 & 0 & 0 & 0 & 0 & 2 & 2 & 5 & 1 & 1 & 2 & 2 & 1 & 3 & 3 & 63 & 21 \\
ME & 0 & 0 & 0 & 0 & 0 & 0 & 4 & 0 & 0 & 1 & 0 & 0 & 0 & 0 & 0 & 0 & 1 & 30 & 10 \\
MI & 0 & 0 & 0 & 0 & 0 & 0 & 2 & 8 & 5 & 17 & 3 & 4 & 5 & 4 & 4 & 5 & 14 & 252 & 146 \\
MN & 0 & 2 & 0 & 0 & 0 & 0 & 7 & 19 & 15 & 61 & 8 & 24 & 19 & 11 & 17 & 27 & 54 & 954 & 698 \\
MO & 0 & 11 & 0 & 0 & 0 & 0 & 11 & 14 & 11 & 49 & 12 & 5 & 9 & 9 & 11 & 20 & 27 & 537 & 413 \\
MP & 0 & 0 & 0 & 0 & 0 & 0 & 0 & 0 & 0 & 0 & 0 & 0 & 0 & 0 & 0 & 0 & 0 & 0 & 0 \\
MS & 0 & 1 & 0 & 0 & 0 & 0 & 2 & 12 & 10 & 21 & 3 & 4 & 6 & 5 & 5 & 8 & 9 & 289 & 158 \\
MT & 0 & 32 & 0 & 0 & -0 & 0 & -3 & 20 & 10 & 32 & 6 & 4 & 7 & 11 & 10 & 8 & 0 & 463 & 615 \\
NC & 0 & 389 & 0 & 0 & 0 & 0 & -86 & 12 & 4 & 16 & 2 & 2 & 4 & 3 & 2 & 8 & 0 & 279 & 242 \\
ND & 0 & 12 & 0 & 0 & 0 & 0 & -55 & 45 & 11 & 72 & 7 & 7 & 12 & 10 & 13 & 17 & 0 & 1,019 & 1,428 \\
NE & 0 & 11 & 0 & 0 & 0 & 0 & 41 & 46 & 46 & 109 & 72 & 74 & 86 & 83 & 73 & 85 & 0 & 708 & 538 \\
NH & 0 & 0 & 0 & 0 & 0 & 0 & 5 & 0 & 0 & 0 & 0 & 0 & 0 & 0 & 0 & 0 & 0 & 7 & 3 \\
NJ & 0 & 0 & 0 & 0 & 0 & -0 & -0 & 0 & 0 & 1 & 0 & 0 & 0 & 0 & 0 & 0 & 0 & 20 & 13 \\
NM & 0 & 17 & 0 & 0 & 0 & -0 & 4 & 13 & 4 & 12 & 11 & 17 & 4 & 10 & 5 & 5 & 0 & 145 & 163 \\
NV & 0 & 1 & 0 & 0 & 0 & -0 & -0 & 2 & 0 & 0 & 1 & 2 & 1 & 1 & 0 & 1 & 0 & 29 & 39 \\
NY & 0 & 0 & 0 & 0 & -0 & 0 & -1 & 2 & 3 & 8 & 1 & 2 & 2 & 3 & 2 & 4 & 0 & 169 & 62 \\
OH & 0 & 12 & 0 & 0 & -0 & 0 & -6 & 9 & 10 & 31 & 2 & 7 & 11 & 9 & 9 & 15 & -0 & 368 & 141 \\
OK & 0 & 34 & 0 & 0 & 0 & 0 & 12 & 21 & 14 & 45 & 33 & 28 & 10 & 9 & 11 & 14 & 0 & 410 & 517 \\
OR & 0 & 3 & 0 & 0 & -0 & -0 & 26 & 26 & 26 & 33 & 17 & 28 & 29 & 30 & 17 & 18 & 0 & 209 & 218 \\
PA & 0 & 0 & 0 & 0 & 0 & -0 & -1 & 1 & 3 & 8 & 1 & 1 & 1 & 1 & 1 & 3 & 0 & 166 & 78 \\
PR & 0 & 0 & 0 & 0 & 0 & 0 & 9 & 0 & 0 & 1 & 0 & 0 & 0 & 1 & 1 & 3 & 0 & 37 & 4 \\
RI & 0 & 0 & 0 & 0 & -0 & -0 & 0 & 0 & 0 & 0 & 0 & 0 & 0 & 0 & 0 & 0 & 0 & 2 & 2 \\
SC & 0 & 71 & 0 & 0 & 0 & 0 & -13 & 8 & 2 & 5 & 1 & 2 & 3 & 2 & 1 & 4 & 0 & 90 & 65 \\
SD & 0 & 41 & 0 & 0 & -0 & -0 & 30 & 17 & 14 & 41 & 17 & 15 & 15 & 18 & 22 & 21 & 0 & 664 & 989 \\
TN & 0 & 73 & 0 & 0 & -0 & 0 & -9 & 18 & 5 & 11 & 2 & 1 & 4 & 5 & 4 & 14 & 0 & 165 & 62 \\
TX & 0 & 27 & 0 & 0 & 0 & 0 & 36 & 148 & 38 & 149 & 39 & 30 & 25 & 26 & 24 & 37 & 0 & 1,195 & 1,472 \\
UT & 0 & 12 & 0 & 0 & 0 & 0 & 1 & 1 & 1 & 3 & 35 & 2 & 1 & 1 & 8 & 3 & 0 & 85 & 104 \\
VA & 0 & 64 & 0 & 0 & -0 & 0 & -17 & 4 & 2 & 5 & 0 & 1 & 1 & 2 & 1 & 2 & 0 & 129 & 79 \\
VT & 0 & 0 & 0 & 0 & -0 & -0 & -0 & 0 & 2 & 1 & 0 & 0 & 0 & 0 & 0 & 0 & 0 & 34 & 7 \\
WA & 0 & 2 & 0 & 0 & 0 & 0 & 23 & 19 & 18 & 33 & 7 & 10 & 25 & 17 & 27 & 10 & 0 & 308 & 342 \\
WI & 0 & 7 & 0 & 0 & 0 & -0 & 4 & 9 & 8 & 25 & 2 & 7 & 7 & 3 & 6 & 11 & 0 & 515 & 154 \\
WV & 0 & 2 & 0 & 0 & 0 & -0 & -1 & 0 & 0 & 1 & 0 & 0 & 0 & 0 & 0 & 0 & 0 & 19 & 7 \\
WY & 0 & 16 & 0 & 0 & -0 & -0 & 3 & 3 & 1 & 4 & 7 & 3 & 1 & 2 & 1 & 1 & 0 & 96 & 147 \\
\end{longtable}

\begin{longtable}{lrrrrrrrrrrrrrrrrrrr}
\toprule
year & 2004 & 2005 & 2006 & 2007 & 2008 & 2009 & 2010 & 2011 & 2012 & 2013 & 2014 & 2015 & 2016 & 2017 & 2018 & 2019 & 2020 & 2021 & 2022 \\
programName &  &  &  &  &  &  &  &  &  &  &  &  &  &  &  &  &  &  &  \\
\midrule
\endfirsthead
\toprule
year & 2004 & 2005 & 2006 & 2007 & 2008 & 2009 & 2010 & 2011 & 2012 & 2013 & 2014 & 2015 & 2016 & 2017 & 2018 & 2019 & 2020 & 2021 & 2022 \\
programName &  &  &  &  &  &  &  &  &  &  &  &  &  &  &  &  &  &  &  \\
\midrule
\endhead
\midrule
\multicolumn{20}{r}{Continued on next page} \\
\midrule
\endfoot
\bottomrule
\endlastfoot
 LIVESTOCK COMPENSATION PROGRAM AUTHORIZE & NaN & NaN & NaN & NaN & NaN & NaN & NaN & NaN & NaN & NaN & NaN & NaN & NaN & NaN & NaN & NaN & NaN & NaN & NaN \\
01-02 CROP DISASTER ASSISTANCE PROGRAM & NaN & NaN & \$ 0.00 & \$ 0.00 & \$ 0.00 & \$ -0.01 & \$ 0.00 & \$ 0.00 & \$ 0.00 & \$ 0.00 & \$ 0.00 & \$ 0.00 & \$ 0.00 & \$ 0.00 & \$ 0.00 & \$ 0.00 & \$ 0.00 & \$ 0.00 & NaN \\
05 - 07 CROP DISASTER ASSISTANCE & NaN & NaN & \$ 0.00 & \$ 0.00 & \$ 0.00 & \$ 0.00 & \$ -546.50 & \$ 0.00 & \$ 0.00 & \$ 0.00 & \$ 0.00 & \$ 0.00 & \$ 0.00 & \$ 0.00 & \$ 0.00 & \$ 0.00 & \$ 0.00 & \$ 0.00 & NaN \\
05 - 07 DAIRY DISASTER PROG & NaN & NaN & \$ 0.00 & \$ 0.00 & \$ 0.00 & \$ -0.07 & \$ 0.00 & \$ 0.00 & \$ 0.00 & \$ 0.00 & \$ 0.00 & \$ 0.00 & \$ 0.00 & \$ 0.00 & \$ 0.00 & \$ 0.00 & \$ 0.00 & \$ 0.00 & NaN \\
05 - 07 LIVESTOCK COMPENSATION PROGRAM & NaN & NaN & \$ 0.00 & \$ 0.00 & \$ 0.00 & \$ 0.00 & \$ 0.00 & \$ 0.00 & \$ 0.00 & \$ 0.00 & \$ 0.00 & \$ 0.00 & \$ 0.00 & \$ 0.00 & \$ 0.00 & \$ 0.00 & \$ 0.00 & \$ 0.00 & NaN \\
05 - 07 LIVESTOCK INDEMNITY PROGRAM & NaN & NaN & \$ 0.00 & \$ 0.00 & \$ 0.00 & \$ 0.00 & \$ 0.00 & \$ 0.00 & \$ 0.00 & \$ 0.00 & \$ 0.00 & \$ 0.00 & \$ 0.00 & \$ 0.00 & \$ 0.00 & \$ 0.00 & \$ 0.00 & \$ 0.00 & NaN \\
90 DAY RULE PAYMENTS & NaN & NaN & \$ 0.00 & \$ 0.00 & \$ 0.38 & \$ 0.00 & \$ -24.06 & \$ 0.00 & \$ 0.00 & \$ 0.00 & \$ 0.00 & \$ 0.00 & \$ 0.00 & \$ 0.00 & \$ 0.00 & \$ 0.00 & \$ 0.00 & \$ 0.00 & NaN \\
ACRE DIRECT PAYMENTS & NaN & NaN & \$ 0.00 & \$ 0.00 & \$ 0.00 & \$ 0.00 & \$ -1,268.47 & \$ 0.00 & \$ 0.00 & \$ 0.00 & \$ 0.00 & \$ 0.00 & \$ 0.00 & \$ 0.00 & \$ 0.00 & \$ 0.00 & \$ 0.00 & \$ 0.00 & NaN \\
ACRE PAYMENTS & NaN & NaN & NaN & NaN & NaN & NaN & NaN & NaN & NaN & NaN & NaN & NaN & NaN & NaN & NaN & NaN & NaN & NaN & NaN \\
ACREAGE GRAZING PAYMENTS & NaN & NaN & NaN & NaN & NaN & NaN & NaN & NaN & NaN & NaN & NaN & NaN & NaN & NaN & NaN & NaN & NaN & NaN & NaN \\
ACREAGE GRAZING PAYMENTS - TRITICALE & NaN & NaN & \$ 0.00 & \$ 0.00 & \$ 0.00 & \$ 0.00 & \$ 0.00 & \$ 0.00 & \$ 0.00 & \$ 0.00 & \$ 0.00 & \$ 0.00 & \$ -0.01 & \$ 245.24 & \$ 0.00 & \$ 0.00 & \$ 0.00 & \$ 0.00 & NaN \\
ACREAGE GRAZING PAYMENTS - WHEAT & NaN & NaN & \$ 0.00 & \$ 0.00 & \$ 0.00 & \$ 0.00 & \$ 0.00 & \$ 0.00 & \$ 0.00 & \$ 0.00 & \$ 0.00 & \$ 0.00 & \$ 7.12 & \$ 34.34 & \$ 0.00 & \$ 0.00 & \$ 0.00 & \$ 0.00 & NaN \\
ADDITIONAL INTEREST PENALTY & NaN & NaN & \$ 0.00 & \$ 0.00 & \$ 0.03 & \$ 0.02 & \$ -309.99 & \$ 0.00 & \$ 0.00 & \$ 0.00 & \$ 0.00 & \$ 0.00 & \$ 0.00 & \$ 0.00 & \$ 0.00 & \$ 0.00 & \$ 0.00 & \$ 0.00 & NaN \\
AGRICULTURAL RISK COVERAGE - INDIVIDUAL & NaN & NaN & \$ 0.00 & \$ 0.00 & \$ 0.00 & \$ 0.00 & \$ 0.00 & \$ 0.00 & \$ 0.00 & \$ 0.00 & \$ 0.00 & \$ -1,258.88 & \$ -628.21 & \$ -435.33 & \$ 55.11 & \$ -978.11 & \$ 131.03 & \$ -433.41 & NaN \\
AGRICULTURAL RISK COVERAGE - INDIVIDUAL       & NaN & NaN & NaN & NaN & NaN & NaN & NaN & NaN & NaN & NaN & NaN & NaN & NaN & NaN & NaN & NaN & NaN & NaN & NaN \\
AGRICULTURAL RISK COVERAGE -COUNTY PILOT & NaN & NaN & \$ 0.00 & \$ 0.00 & \$ 0.00 & \$ 0.00 & \$ 0.00 & \$ 0.00 & \$ 0.00 & \$ 0.00 & \$ 0.00 & \$ 0.00 & \$ 0.00 & \$ 0.00 & \$ -1,900.86 & \$ -1,066.69 & \$ -23.00 & \$ 173.29 & NaN \\
AGRICULTURAL RISK COVERAGE PROG - COUNTY & NaN & NaN & \$ 0.00 & \$ 0.00 & \$ 0.00 & \$ 0.00 & \$ 0.00 & \$ 0.00 & \$ 0.00 & \$ 0.00 & \$ 0.00 & \$ -114.65 & \$ 10.13 & \$ -57.05 & \$ -24.93 & \$ 262.48 & \$ 25.22 & \$ -472.22 & NaN \\
AGRICULTURAL RISK COVERAGE PROG - COUNTY      & NaN & NaN & NaN & NaN & NaN & NaN & NaN & NaN & NaN & NaN & NaN & NaN & NaN & NaN & NaN & NaN & NaN & NaN & NaN \\
AMA ORGANIC COST SHARE - CROPS & NaN & NaN & \$ 0.00 & \$ 0.00 & \$ 0.00 & \$ 0.00 & \$ 0.00 & \$ 0.00 & \$ 0.00 & \$ 0.00 & \$ 0.00 & \$ 0.00 & \$ 0.00 & \$ -5.16 & \$ 4.89 & \$ 8.81 & \$ 0.00 & \$ 0.00 & NaN \\
AMA ORGANIC COST SHARE - LIVESTOCK & NaN & NaN & \$ 0.00 & \$ 0.00 & \$ 0.00 & \$ 0.00 & \$ 0.00 & \$ 0.00 & \$ 0.00 & \$ 0.00 & \$ 0.00 & \$ 0.00 & \$ 0.00 & \$ 0.00 & \$ -22.28 & \$ -33.80 & \$ 0.00 & \$ 0.00 & NaN \\
AMA ORGANIC COST SHARE - WILD CROPS & NaN & NaN & \$ 0.00 & \$ 0.00 & \$ 0.00 & \$ 0.00 & \$ 0.00 & \$ 0.00 & \$ 0.00 & \$ 0.00 & \$ 0.00 & \$ 0.00 & \$ 0.00 & \$ 0.00 & \$ 0.00 & \$ 0.00 & \$ 0.00 & \$ 0.00 & NaN \\
ARC PROGRAM INDIVIDUAL COVERAGE & NaN & NaN & NaN & NaN & NaN & NaN & NaN & NaN & NaN & NaN & NaN & NaN & NaN & NaN & NaN & NaN & NaN & NaN & NaN \\
ARC PROGRAM-COUNTY COVERAGE & NaN & NaN & NaN & NaN & NaN & NaN & NaN & NaN & NaN & NaN & NaN & NaN & NaN & NaN & NaN & NaN & NaN & NaN & NaN \\
ARC PROGRAM-COUNTY PILOT COVERAGE & NaN & NaN & NaN & NaN & NaN & NaN & NaN & NaN & NaN & NaN & NaN & NaN & NaN & NaN & NaN & NaN & NaN & NaN & NaN \\
AUTOMATED CONSERVATION RESERVE PROGRAM - COST SHARE & NaN & NaN & \$ 0.00 & \$ 0.00 & \$ 0.00 & \$ 0.00 & \$ -1,125.16 & \$ 0.00 & \$ 0.00 & \$ 0.00 & \$ 0.00 & \$ 0.00 & \$ 0.00 & \$ 0.00 & \$ 0.00 & \$ 0.00 & \$ 0.00 & \$ 0.00 & NaN \\
AVERAGE CROP REVENUE ELECTION - DIRECT & NaN & NaN & \$ 0.00 & \$ 0.00 & \$ 0.00 & \$ 0.00 & \$ 0.00 & \$ 0.00 & \$ 0.00 & \$ 0.00 & \$ 0.00 & \$ 0.00 & \$ 0.00 & \$ -499.43 & \$ 0.00 & \$ 0.00 & \$ 0.00 & \$ 0.00 & NaN \\
AVERAGE CROP REVENUE ELECTION-ACRE & NaN & NaN & \$ 0.00 & \$ 0.00 & \$ 0.00 & \$ 0.00 & \$ 4,329.85 & \$ 521.52 & \$ 37.84 & \$ -19.29 & \$ 18.96 & \$ 192.57 & \$ 2,121.37 & \$ 5,597.96 & \$ 0.00 & \$ 0.00 & \$ 0.00 & \$ 0.00 & NaN \\
AVERAGE CROP REVENUE ELECTION-ACRE            & NaN & NaN & NaN & NaN & NaN & NaN & NaN & NaN & NaN & NaN & NaN & NaN & NaN & NaN & NaN & NaN & NaN & NaN & NaN \\
AVERAGE CROP REVENUE ELECTION-DIRECT & NaN & NaN & \$ 0.00 & \$ 0.00 & \$ 0.00 & \$ 0.00 & \$ 1,259.23 & \$ -0.88 & \$ -65.27 & \$ 4.56 & \$ 522.45 & \$ 790.80 & \$ 382.63 & \$ 61.68 & \$ 0.00 & \$ 0.00 & \$ 0.00 & \$ 0.00 & NaN \\
AVERAGE CROP REVENUE ELECTION-DIRECT          & NaN & NaN & NaN & NaN & NaN & NaN & NaN & NaN & NaN & NaN & NaN & NaN & NaN & NaN & NaN & NaN & NaN & NaN & NaN \\
BCAP ANNUAL RENTAL WEB BASED & NaN & NaN & NaN & NaN & NaN & NaN & NaN & NaN & NaN & NaN & NaN & NaN & NaN & NaN & NaN & NaN & NaN & NaN & NaN \\
BCAP COST SHARE WEB BASED & NaN & NaN & NaN & NaN & NaN & NaN & NaN & NaN & NaN & NaN & NaN & NaN & NaN & NaN & NaN & NaN & NaN & NaN & NaN \\
BCAP MATCHING PYMNT WEB-BASE APPLICATION & NaN & NaN & NaN & NaN & NaN & NaN & NaN & NaN & NaN & NaN & NaN & NaN & NaN & NaN & NaN & NaN & NaN & NaN & NaN \\
BCAP PRIVATE SECTOR TECHNICAL ASSISTANCE & NaN & NaN & NaN & NaN & NaN & NaN & NaN & NaN & NaN & NaN & NaN & NaN & NaN & NaN & NaN & NaN & NaN & NaN & NaN \\
BIOFUEL INFRASTRUCTURE PROGRAM & NaN & NaN & NaN & NaN & NaN & NaN & NaN & NaN & NaN & NaN & NaN & NaN & NaN & NaN & NaN & NaN & NaN & NaN & NaN \\
BIOMASS CROP ASSIST COLLECTION MATCH PAY & NaN & NaN & \$ 0.00 & \$ 0.00 & \$ 0.00 & \$ 0.00 & \$ 0.00 & \$ -13,710.09 & \$ -12,919.77 & \$ 0.00 & \$ -24,235.42 & \$ -13,002.19 & \$ -10,940.92 & \$ -9,421.16 & \$ -30,366.00 & \$ 0.00 & \$ 0.00 & \$ 0.00 & NaN \\
BIOMASS CROP ASSIST COLLECTION MATCH PAY      & NaN & NaN & NaN & NaN & NaN & NaN & NaN & NaN & NaN & NaN & NaN & NaN & NaN & NaN & NaN & NaN & NaN & NaN & NaN \\
BIOMASS CROP ASSISTANCE & NaN & NaN & \$ 0.00 & \$ 0.00 & \$ 0.00 & \$ 0.00 & \$ -8,597.89 & \$ 0.00 & \$ 0.00 & \$ 0.00 & \$ 0.00 & \$ 0.00 & \$ 0.00 & \$ 0.00 & \$ 0.00 & \$ 0.00 & \$ 0.00 & \$ 0.00 & NaN \\
BIOMASS CROP ASSISTANCE - ANNUAL RENTAL & NaN & NaN & \$ 0.00 & \$ 0.00 & \$ 0.00 & \$ 0.00 & \$ 0.00 & \$ -1,676.90 & \$ -2,819.39 & \$ -4,117.75 & \$ -4,221.19 & \$ -4,165.45 & \$ -3,482.23 & \$ -2,797.07 & \$ -1,721.22 & \$ -2,351.42 & \$ 0.00 & \$ 0.00 & NaN \\
BIOMASS CROP ASSISTANCE - ANNUAL RENTAL       & NaN & NaN & NaN & NaN & NaN & NaN & NaN & NaN & NaN & NaN & NaN & NaN & NaN & NaN & NaN & NaN & NaN & NaN & NaN \\
BIOMASS CROP ASSISTANCE - COST-SHARE & NaN & NaN & NaN & NaN & NaN & NaN & NaN & NaN & NaN & NaN & NaN & NaN & NaN & NaN & NaN & NaN & NaN & NaN & NaN \\
BIOMASS CROP ASSISTANCE - COST-SHARE          & NaN & NaN & NaN & NaN & NaN & NaN & NaN & NaN & NaN & NaN & NaN & NaN & NaN & NaN & NaN & NaN & NaN & NaN & NaN \\
BIOMASS CROP ASSISTANCE TITLE 1 CROP RESIDUE & NaN & NaN & \$ 0.00 & \$ 0.00 & \$ 0.00 & \$ 0.00 & \$ -9,070.90 & \$ 0.00 & \$ 0.00 & \$ 0.00 & \$ 0.00 & \$ 0.00 & \$ 0.00 & \$ 0.00 & \$ 0.00 & \$ 0.00 & \$ 0.00 & \$ 0.00 & NaN \\
CARES ACT (OLP) & NaN & NaN & NaN & NaN & NaN & NaN & NaN & NaN & NaN & NaN & NaN & NaN & NaN & NaN & NaN & NaN & NaN & NaN & NaN \\
CCC ORGANIC COST SHARE - CROPS & NaN & NaN & \$ 0.00 & \$ 0.00 & \$ 0.00 & \$ 0.00 & \$ 0.00 & \$ 0.00 & \$ 0.00 & \$ 0.00 & \$ 0.00 & \$ 0.00 & \$ 0.00 & \$ 1.60 & \$ 1.23 & \$ 0.20 & \$ 0.02 & \$ 1.15 & NaN \\
CCC ORGANIC COST SHARE - LIVESTOCK & NaN & NaN & \$ 0.00 & \$ 0.00 & \$ 0.00 & \$ 0.00 & \$ 0.00 & \$ 0.00 & \$ 0.00 & \$ 0.00 & \$ 0.00 & \$ 0.00 & \$ 0.00 & \$ -2.82 & \$ 0.18 & \$ -0.15 & \$ 1.28 & \$ -8.12 & NaN \\
CCC ORGANIC COST SHARE - WILD CROPS & NaN & NaN & \$ 0.00 & \$ 0.00 & \$ 0.00 & \$ 0.00 & \$ 0.00 & \$ 0.00 & \$ 0.00 & \$ 0.00 & \$ 0.00 & \$ 0.00 & \$ 0.00 & \$ 0.00 & \$ 0.00 & \$ -1.75 & \$ 15.00 & \$ -144.39 & NaN \\
CCC ORGANIC COST SHARE FEES - HANDLING & NaN & NaN & \$ 0.00 & \$ 0.00 & \$ 0.00 & \$ 0.00 & \$ 0.00 & \$ 0.00 & \$ 0.00 & \$ 0.00 & \$ 0.00 & \$ 0.00 & \$ 0.00 & \$ -0.08 & \$ 23.98 & \$ 1.84 & \$ 0.98 & \$ 29.34 & NaN \\
CFAP3 - LTU & NaN & NaN & NaN & NaN & NaN & NaN & NaN & NaN & NaN & NaN & NaN & NaN & NaN & NaN & NaN & NaN & NaN & NaN & NaN \\
CFAP3 - TUP & NaN & NaN & NaN & NaN & NaN & NaN & NaN & NaN & NaN & NaN & NaN & NaN & NaN & NaN & NaN & NaN & NaN & NaN & NaN \\
CFAPCARES & NaN & NaN & \$ 0.00 & \$ 0.00 & \$ 0.00 & \$ 0.00 & \$ 0.00 & \$ 0.00 & \$ 0.00 & \$ 0.00 & \$ 0.00 & \$ 0.00 & \$ 0.00 & \$ 0.00 & \$ 0.00 & \$ 0.00 & \$ 23.99 & \$ 8,420.05 & NaN \\
CFAPCARES2 & NaN & NaN & \$ 0.00 & \$ 0.00 & \$ 0.00 & \$ 0.00 & \$ 0.00 & \$ 0.00 & \$ 0.00 & \$ 0.00 & \$ 0.00 & \$ 0.00 & \$ 0.00 & \$ 0.00 & \$ 0.00 & \$ 0.00 & \$ -23.21 & \$ -21,709.70 & NaN \\
CFAPCARES2.1 & NaN & NaN & NaN & NaN & NaN & NaN & NaN & NaN & NaN & NaN & NaN & NaN & NaN & NaN & NaN & NaN & NaN & NaN & NaN \\
CFAPCCA2 & NaN & NaN & \$ 0.00 & \$ 0.00 & \$ 0.00 & \$ 0.00 & \$ 0.00 & \$ 0.00 & \$ 0.00 & \$ 0.00 & \$ 0.00 & \$ 0.00 & \$ 0.00 & \$ 0.00 & \$ 0.00 & \$ 0.00 & \$ -212.49 & \$ -4,416.84 & NaN \\
CFAPCCCCA & NaN & NaN & \$ 0.00 & \$ 0.00 & \$ 0.00 & \$ 0.00 & \$ 0.00 & \$ 0.00 & \$ 0.00 & \$ 0.00 & \$ 0.00 & \$ 0.00 & \$ 0.00 & \$ 0.00 & \$ 0.00 & \$ 0.00 & \$ 18.64 & \$ 2,397.19 & NaN \\
CHESAPEAKE BAY INCENTIVE & NaN & NaN & NaN & NaN & NaN & NaN & NaN & NaN & NaN & NaN & NaN & NaN & NaN & NaN & NaN & NaN & NaN & NaN & NaN \\
CONSERVATION - EMERGENCY & NaN & NaN & \$ 0.00 & \$ 0.00 & \$ 0.00 & \$ 0.00 & \$ 1,353.55 & \$ -17.06 & \$ 462.23 & \$ 597.40 & \$ 0.00 & \$ 0.00 & \$ 0.00 & \$ 0.00 & \$ 0.00 & \$ 0.00 & \$ 0.00 & \$ 0.00 & NaN \\
CONSERVATION RESERVE PROGRAM - ANNUAL LAND RENTAL & NaN & NaN & \$ 0.00 & \$ 0.00 & \$ 0.00 & \$ 0.00 & \$ -1,260.59 & \$ 0.00 & \$ 0.00 & \$ 0.00 & \$ 0.00 & \$ 0.00 & \$ 0.00 & \$ 0.00 & \$ 0.00 & \$ 0.00 & \$ 0.00 & \$ 0.00 & NaN \\
CONSERVATION RESERVE PROGRAM CANCELLATION, MISACTION & NaN & NaN & \$ 0.00 & \$ 0.00 & \$ 0.10 & \$ 0.21 & \$ -954.50 & \$ 0.00 & \$ 0.00 & \$ 0.00 & \$ 0.00 & \$ 0.00 & \$ 0.00 & \$ 0.00 & \$ 0.00 & \$ 0.00 & \$ 0.00 & \$ 0.00 & NaN \\
CORONAVIRUS AID, CCC CHARTER ACT (OLP) & NaN & NaN & NaN & NaN & NaN & NaN & NaN & NaN & NaN & NaN & NaN & NaN & NaN & NaN & NaN & NaN & NaN & NaN & NaN \\
COTTON GINNING COST SHARE & NaN & NaN & NaN & NaN & NaN & NaN & NaN & NaN & NaN & NaN & NaN & NaN & NaN & NaN & NaN & NaN & NaN & NaN & NaN \\
COTTON GINNING COST SHARE PROGRAM & NaN & NaN & \$ 0.00 & \$ 0.00 & \$ 0.00 & \$ 0.00 & \$ 0.00 & \$ 0.00 & \$ 0.00 & \$ 0.00 & \$ 0.00 & \$ 0.00 & \$ 1,142.78 & \$ 11.24 & \$ 34,323.11 & \$ -738.00 & \$ 0.00 & \$ 0.00 & NaN \\
COTTON GINNING COST SHARE PROGRAM             & NaN & NaN & NaN & NaN & NaN & NaN & NaN & NaN & NaN & NaN & NaN & NaN & NaN & NaN & NaN & NaN & NaN & NaN & NaN \\
COTTON TRANSITION ASSISTANCE PROGRAM & NaN & NaN & \$ 0.00 & \$ 0.00 & \$ 0.00 & \$ 0.00 & \$ 0.00 & \$ 0.00 & \$ 0.00 & \$ 0.00 & \$ 9.41 & \$ 31.01 & \$ 430.03 & \$ 386.30 & \$ 0.00 & \$ 0.00 & \$ 0.00 & \$ 0.00 & NaN \\
COTTON TRANSITION ASSISTANCE PROGRAM          & NaN & NaN & NaN & NaN & NaN & NaN & NaN & NaN & NaN & NaN & NaN & NaN & NaN & NaN & NaN & NaN & NaN & NaN & NaN \\
COUNTER CYCLICAL PAYMENT - UPLAND COTTON & NaN & NaN & NaN & NaN & NaN & NaN & NaN & NaN & NaN & NaN & NaN & NaN & NaN & NaN & NaN & NaN & NaN & NaN & NaN \\
COUNTER CYCLICAL PYMTS FROM SYS/36 & NaN & NaN & NaN & NaN & NaN & NaN & NaN & NaN & NaN & NaN & NaN & NaN & NaN & NaN & NaN & NaN & NaN & NaN & NaN \\
CROP ASSISTANCE PROGRAM & NaN & NaN & \$ 0.00 & \$ 0.00 & \$ 0.00 & \$ 0.00 & \$ 2,915.80 & \$ 0.09 & \$ -350.95 & \$ -0.21 & \$ 0.00 & \$ 0.00 & \$ 0.00 & \$ 0.00 & \$ 0.00 & \$ 0.00 & \$ 0.00 & \$ 0.00 & NaN \\
CROP DISASTER ASSISTANCE PROG AUTHORIZED & NaN & NaN & \$ 0.00 & \$ 0.00 & \$ 0.00 & \$ 0.01 & \$ 577.61 & \$ 971.37 & \$ 0.00 & \$ 0.00 & \$ 0.00 & \$ 0.00 & \$ 0.00 & \$ 0.00 & \$ 0.00 & \$ 0.00 & \$ 0.00 & \$ 0.00 & NaN \\
CROP DISASTER PROGRAM & NaN & NaN & \$ 0.00 & \$ 0.00 & \$ 0.00 & \$ 0.00 & \$ -2,270.25 & \$ -675.17 & \$ -3,747.70 & \$ 0.00 & \$ 0.00 & \$ 0.00 & \$ 0.00 & \$ 0.00 & \$ 0.00 & \$ 0.00 & \$ 0.00 & \$ 0.00 & NaN \\
CRP - CHESAPEAKE BAY INCENTIVE & NaN & NaN & \$ 0.00 & \$ 0.00 & \$ 0.00 & \$ 0.00 & \$ 0.00 & \$ 0.00 & \$ 0.00 & \$ 0.00 & \$ 0.00 & \$ 0.00 & \$ -4,032.75 & \$ -2,462.58 & \$ -1,481.17 & \$ -2,944.13 & \$ 0.00 & \$ 1,515.88 & NaN \\
CRP - CHESAPEAKE BAY INCENTIVE                & NaN & NaN & NaN & NaN & NaN & NaN & NaN & NaN & NaN & NaN & NaN & NaN & NaN & NaN & NaN & NaN & NaN & NaN & NaN \\
CRP - CONTINUOUS PIP & NaN & NaN & \$ 0.00 & \$ 0.00 & \$ 0.00 & \$ 0.00 & \$ 0.00 & \$ 0.00 & \$ 0.00 & \$ 0.00 & \$ 0.00 & \$ 0.00 & \$ 0.00 & \$ 0.00 & \$ 0.00 & \$ 0.00 & \$ 0.00 & \$ 673.68 & NaN \\
CRP - EMERGENCY FORESTRY ANNUAL RENTAL & NaN & NaN & \$ 0.00 & \$ 0.00 & \$ 0.00 & \$ 0.00 & \$ 210.85 & \$ 29.94 & \$ 39.00 & \$ 45.31 & \$ 27.85 & \$ 25.30 & \$ 198.02 & \$ 45.27 & \$ 50.70 & \$ 15.88 & \$ 7.11 & \$ -3,020.67 & NaN \\
CRP - EMERGENCY FORESTRY ANNUAL RENTAL        & NaN & NaN & NaN & NaN & NaN & NaN & NaN & NaN & NaN & NaN & NaN & NaN & NaN & NaN & NaN & NaN & NaN & NaN & NaN \\
CRP - EMERGENCY FORESTRY COST SHARE & NaN & NaN & \$ 0.00 & \$ 0.00 & \$ 0.00 & \$ 0.00 & \$ -12.37 & \$ 21.52 & \$ 24.19 & \$ 324.92 & \$ 101.08 & \$ 0.00 & \$ -80.31 & \$ 0.00 & \$ 0.00 & \$ 0.00 & \$ 0.00 & \$ 0.00 & NaN \\
CRP - EMERGENCY FORESTRY COST SHARE           & NaN & NaN & NaN & NaN & NaN & NaN & NaN & NaN & NaN & NaN & NaN & NaN & NaN & NaN & NaN & NaN & NaN & NaN & NaN \\
CRP - PRACTICE INCENTIVE & NaN & NaN & \$ 0.00 & \$ 0.00 & \$ -0.00 & \$ -0.00 & \$ -2,137.96 & \$ 0.00 & \$ 0.00 & \$ 0.00 & \$ 0.00 & \$ 0.00 & \$ 0.00 & \$ 0.00 & \$ 0.00 & \$ 0.00 & \$ 0.00 & \$ 0.00 & NaN \\
CRP - RIPARIAN BUFFER INCENTIVE & NaN & NaN & \$ 0.00 & \$ 0.00 & \$ 0.01 & \$ 0.04 & \$ -6,981.54 & \$ 0.00 & \$ 0.00 & \$ 0.00 & \$ 0.00 & \$ 0.00 & \$ 0.00 & \$ 0.00 & \$ 0.00 & \$ 0.00 & \$ 0.00 & \$ 0.00 & NaN \\
CRP - SIGNING INCENTIVE & NaN & NaN & \$ 0.00 & \$ 0.00 & \$ 0.00 & \$ 0.00 & \$ -1,677.50 & \$ 0.00 & \$ 0.00 & \$ 0.00 & \$ 0.00 & \$ 0.00 & \$ 0.00 & \$ 0.00 & \$ 0.00 & \$ 0.00 & \$ 0.00 & \$ 0.00 & NaN \\
CRP - TREE THINNING INCENTIVE PROGRAM & NaN & NaN & \$ 0.00 & \$ 0.00 & \$ 0.00 & \$ 0.00 & \$ 0.00 & \$ 0.00 & \$ 0.00 & \$ 0.00 & \$ 0.00 & \$ 0.00 & \$ 0.00 & \$ 192.19 & \$ -114.81 & \$ -1,094.91 & \$ 489.89 & \$ 9,246.00 & NaN \\
CRP ANNUAL RENTAL & NaN & NaN & \$ 0.00 & \$ 0.00 & \$ 0.00 & \$ 0.02 & \$ -469.80 & \$ 0.00 & \$ 0.00 & \$ 0.00 & \$ 0.00 & \$ 0.00 & \$ 0.00 & \$ 0.00 & \$ 0.00 & \$ 0.00 & \$ 0.00 & \$ 0.00 & NaN \\
CRP BIOMASS CROP ASSISTANCE PROGRAM & NaN & NaN & \$ 0.00 & \$ 0.00 & \$ 0.00 & \$ 0.00 & \$ 8,727.40 & \$ -4,411.68 & \$ 89,332.38 & \$ 0.00 & \$ 0.00 & \$ 0.00 & \$ 0.00 & \$ 0.00 & \$ 0.00 & \$ 0.00 & \$ 0.00 & \$ 0.00 & NaN \\
CRP CHESAPEAKE BAY INCENTIVE & NaN & NaN & NaN & NaN & NaN & NaN & NaN & NaN & NaN & NaN & NaN & NaN & NaN & NaN & NaN & NaN & NaN & NaN & NaN \\
CRP CLEAR30 MAINT PRGM & NaN & NaN & NaN & NaN & NaN & NaN & NaN & NaN & NaN & NaN & NaN & NaN & NaN & NaN & NaN & NaN & NaN & NaN & NaN \\
CRP COST-SHARE WEB-BASED - COF & NaN & NaN & \$ 0.00 & \$ 0.00 & \$ 0.00 & \$ 0.00 & \$ 0.00 & \$ 0.00 & \$ 0.00 & \$ 0.00 & \$ 0.00 & \$ 0.00 & \$ 0.00 & \$ 0.00 & \$ 0.00 & \$ 0.00 & \$ 0.00 & \$ 455.33 & NaN \\
CRP FOREST INVENTORY PILOT PROGRAM & NaN & NaN & NaN & NaN & NaN & NaN & NaN & NaN & NaN & NaN & NaN & NaN & NaN & NaN & NaN & NaN & NaN & NaN & NaN \\
CRP FOREST MANAGEMENT INCENTIVE & NaN & NaN & NaN & NaN & NaN & NaN & NaN & NaN & NaN & NaN & NaN & NaN & NaN & NaN & NaN & NaN & NaN & NaN & NaN \\
CRP HONEY BEE INCENTIVE PAYMNTS & NaN & NaN & \$ 0.00 & \$ 0.00 & \$ 0.00 & \$ 0.00 & \$ 0.00 & \$ 0.00 & \$ 0.00 & \$ 0.00 & \$ 0.00 & \$ 0.00 & \$ 0.00 & \$ 0.00 & \$ 0.00 & \$ 0.00 & \$ 0.00 & \$ 0.00 & NaN \\
CRP INCENTIVES & NaN & NaN & NaN & NaN & NaN & NaN & NaN & NaN & NaN & NaN & NaN & NaN & NaN & NaN & NaN & NaN & NaN & NaN & NaN \\
CRP PAYMENT - ANNUAL & NaN & NaN & \$ 0.00 & \$ 0.00 & \$ 0.00 & \$ 0.00 & \$ 1,994.16 & \$ 410.61 & \$ 0.00 & \$ 0.00 & \$ 0.00 & \$ 0.00 & \$ 0.00 & \$ 0.00 & \$ 0.00 & \$ 0.00 & \$ 0.00 & \$ 0.00 & NaN \\
CRP PAYMENT - ANNUAL PAYMENT & NaN & NaN & \$ 0.00 & \$ 0.00 & \$ 0.02 & \$ 0.00 & \$ -1,793.11 & \$ 0.00 & \$ 0.00 & \$ 0.00 & \$ 0.00 & \$ 0.00 & \$ 0.00 & \$ 0.00 & \$ 0.00 & \$ 0.00 & \$ 0.00 & \$ 0.00 & NaN \\
CRP PAYMENT - ANNUAL RENTAL & NaN & NaN & \$ 0.00 & \$ 0.00 & \$ 0.00 & \$ 0.00 & \$ 1,811.87 & \$ 28.97 & \$ 20.27 & \$ 22.23 & \$ 22.53 & \$ 30.08 & \$ -84.46 & \$ 27.69 & \$ 30.74 & \$ 30.76 & \$ 4.21 & \$ 634.37 & NaN \\
CRP PAYMENT - ANNUAL RENTAL                   & NaN & NaN & NaN & NaN & NaN & NaN & NaN & NaN & NaN & NaN & NaN & NaN & NaN & NaN & NaN & NaN & NaN & NaN & NaN \\
CRP PRACTICE INCENTIVES PAYMENT & NaN & NaN & \$ 0.00 & \$ 0.00 & \$ 0.00 & \$ 0.00 & \$ 0.00 & \$ 0.00 & \$ 0.00 & \$ 0.00 & \$ 0.00 & \$ 0.00 & \$ 0.00 & \$ 0.00 & \$ 0.00 & \$ 0.00 & \$ -0.16 & \$ -54.39 & NaN \\
CRP TRANSITION INCENTIVES PRGM & NaN & NaN & \$ 0.00 & \$ 0.00 & \$ 0.00 & \$ 0.00 & \$ 0.00 & \$ 0.00 & \$ 0.00 & \$ 0.00 & \$ 0.00 & \$ 0.00 & \$ 0.00 & \$ 0.00 & \$ 0.00 & \$ 0.00 & \$ 19.88 & \$ 1,021.97 & NaN \\
CRP WETLAND RESTORATION PROGRAM & NaN & NaN & \$ 0.00 & \$ 0.00 & \$ 0.00 & \$ -0.01 & \$ -3,396.86 & \$ 0.00 & \$ 0.00 & \$ 0.00 & \$ 0.00 & \$ 0.00 & \$ 0.00 & \$ 0.00 & \$ 0.00 & \$ 0.00 & \$ 0.00 & \$ 0.00 & NaN \\
DAIRY ECONOMIC LOSS ASSISTANCE PROGRAM & NaN & NaN & \$ 0.00 & \$ 0.00 & \$ 0.00 & \$ -0.01 & \$ 2.56 & \$ 20.60 & \$ 328.58 & \$ -188.21 & \$ -1,144.56 & \$ 0.00 & \$ 0.00 & \$ 0.00 & \$ 0.00 & \$ 0.00 & \$ 0.00 & \$ 0.00 & NaN \\
DAIRY INDEMNITY PROGRAM & NaN & NaN & \$ 0.00 & \$ 0.00 & \$ -0.21 & \$ -0.03 & \$ -9,927.87 & \$ 0.00 & \$ 0.00 & \$ 0.00 & \$ 0.00 & \$ 0.00 & \$ 0.00 & \$ 0.00 & \$ 0.00 & \$ 0.00 & \$ 0.00 & \$ 0.00 & NaN \\
DAIRY MARGIN COVERAGE & NaN & NaN & \$ 0.00 & \$ 0.00 & \$ 0.00 & \$ 0.00 & \$ 0.00 & \$ 0.00 & \$ 0.00 & \$ 0.00 & \$ 0.00 & \$ 0.00 & \$ 0.00 & \$ 0.00 & \$ 0.00 & \$ 0.00 & \$ 170.39 & \$ -1,885.85 & NaN \\
DAIRY MARGIN COVERAGE PROGRAM & NaN & NaN & \$ 0.00 & \$ 0.00 & \$ 0.00 & \$ 0.00 & \$ 0.00 & \$ 0.00 & \$ 0.00 & \$ 0.00 & \$ 0.00 & \$ 0.00 & \$ 0.00 & \$ 0.00 & \$ 0.00 & \$ 1,211.92 & \$ 1,504.94 & \$ 7,309.38 & NaN \\
DCP - COUNTER CYCLICAL & NaN & NaN & \$ 0.00 & \$ 0.00 & \$ 0.00 & \$ 0.00 & \$ 352.82 & \$ 29.34 & \$ 270.94 & \$ 362.63 & \$ 170.60 & \$ -277.66 & \$ 84.09 & \$ -88.54 & \$ 0.00 & \$ 0.00 & \$ 0.00 & \$ 0.00 & NaN \\
DCP - COUNTER CYCLICAL                        & NaN & NaN & NaN & NaN & NaN & NaN & NaN & NaN & NaN & NaN & NaN & NaN & NaN & NaN & NaN & NaN & NaN & NaN & NaN \\
DCP - DIRECT & NaN & NaN & \$ 0.00 & \$ 0.00 & \$ 0.00 & \$ 0.00 & \$ 1,143.26 & \$ 2.43 & \$ -28.89 & \$ -32.28 & \$ 233.66 & \$ 504.95 & \$ 478.97 & \$ 347.41 & \$ -268.12 & \$ 0.00 & \$ 0.00 & \$ 0.00 & NaN \\
DCP - DIRECT                                  & NaN & NaN & NaN & NaN & NaN & NaN & NaN & NaN & NaN & NaN & NaN & NaN & NaN & NaN & NaN & NaN & NaN & NaN & NaN \\
DCP PROGRAM - COUNTER CYCLICAL PAYMENTS & NaN & NaN & \$ 0.00 & \$ 0.00 & \$ 0.00 & \$ -0.01 & \$ -351.24 & \$ 0.00 & \$ 0.00 & \$ 0.00 & \$ 0.00 & \$ 0.00 & \$ 0.00 & \$ 0.00 & \$ 0.00 & \$ 0.00 & \$ 0.00 & \$ 0.00 & NaN \\
DCP PROGRAM - DIRECT PAYMENTS & NaN & NaN & \$ 0.00 & \$ 0.00 & \$ 0.00 & \$ 0.00 & \$ -1,148.93 & \$ 0.00 & \$ 0.00 & \$ 0.00 & \$ 0.00 & \$ 0.00 & \$ 0.00 & \$ 0.00 & \$ 0.00 & \$ 0.00 & \$ 0.00 & \$ 0.00 & NaN \\
DIPP WEB-BASED & NaN & NaN & \$ 0.00 & \$ 0.00 & \$ 0.00 & \$ 0.00 & \$ 0.00 & \$ 0.00 & \$ 0.00 & \$ 0.00 & \$ 0.00 & \$ 0.00 & \$ 0.00 & \$ 0.00 & \$ 0.00 & \$ 0.00 & \$ 0.00 & \$ 57,722.40 & NaN \\
DIRECT AND COUNTER CYCLICAL PROG & NaN & NaN & \$ 0.00 & \$ 0.00 & \$ -0.00 & \$ 0.00 & \$ -1,497.81 & \$ 0.00 & \$ 0.00 & \$ 0.00 & \$ 0.00 & \$ 0.00 & \$ 0.00 & \$ 0.00 & \$ 0.00 & \$ 0.00 & \$ 0.00 & \$ 0.00 & NaN \\
DIRECT PAYMENT - BARLEY & NaN & NaN & \$ 0.00 & \$ 0.00 & \$ 0.00 & \$ 0.00 & \$ 307.17 & \$ 0.00 & \$ 0.00 & \$ 0.00 & \$ 98.10 & \$ 0.00 & \$ 0.00 & \$ 0.00 & \$ 0.00 & \$ 0.00 & \$ 0.00 & \$ 0.00 & NaN \\
DIRECT PAYMENT - CANOLA SEED & NaN & NaN & NaN & NaN & NaN & NaN & NaN & NaN & NaN & NaN & NaN & NaN & NaN & NaN & NaN & NaN & NaN & NaN & NaN \\
DIRECT PAYMENT - CORN & NaN & NaN & \$ 0.00 & \$ 0.00 & \$ 0.00 & \$ 0.00 & \$ 941.49 & \$ 35.49 & \$ -61.75 & \$ 0.00 & \$ 8.50 & \$ 0.00 & \$ 0.00 & \$ 0.00 & \$ 0.00 & \$ 0.00 & \$ 0.00 & \$ 0.00 & NaN \\
DIRECT PAYMENT - FLAXSEED & NaN & NaN & \$ 0.00 & \$ 0.00 & \$ 0.00 & \$ 0.00 & \$ 61.00 & \$ 0.00 & \$ 0.00 & \$ 0.00 & \$ 0.00 & \$ 0.00 & \$ 0.00 & \$ 0.00 & \$ 0.00 & \$ 0.00 & \$ 0.00 & \$ 0.00 & NaN \\
DIRECT PAYMENT - OATS & NaN & NaN & \$ 0.00 & \$ 0.00 & \$ 0.00 & \$ 0.00 & \$ 8.90 & \$ 0.00 & \$ 0.00 & \$ 0.00 & \$ 0.00 & \$ 0.00 & \$ 0.00 & \$ 0.00 & \$ 0.00 & \$ 0.00 & \$ 0.00 & \$ 0.00 & NaN \\
DIRECT PAYMENT - PEANUTS & NaN & NaN & \$ 0.00 & \$ 0.00 & \$ 0.00 & \$ 0.00 & \$ 2,663.39 & \$ 0.00 & \$ 0.00 & \$ 0.00 & \$ 0.00 & \$ 0.00 & \$ 0.00 & \$ 0.00 & \$ 0.00 & \$ 0.00 & \$ 0.00 & \$ 0.00 & NaN \\
DIRECT PAYMENT - RICE & NaN & NaN & \$ 0.00 & \$ 0.00 & \$ 0.00 & \$ 0.00 & \$ 5,820.37 & \$ -275.87 & \$ 173.60 & \$ 0.00 & \$ -18.42 & \$ 0.00 & \$ 0.00 & \$ 0.00 & \$ 0.00 & \$ 0.00 & \$ 0.00 & \$ 0.00 & NaN \\
DIRECT PAYMENT - SAFFLOWER SEED & NaN & NaN & \$ 0.00 & \$ 0.00 & \$ 0.00 & \$ 0.00 & \$ 160.46 & \$ 0.00 & \$ 0.00 & \$ 0.00 & \$ 0.00 & \$ 0.00 & \$ 0.00 & \$ 0.00 & \$ 0.00 & \$ 0.00 & \$ 0.00 & \$ 0.00 & NaN \\
DIRECT PAYMENT - SESAME & NaN & NaN & NaN & NaN & NaN & NaN & NaN & NaN & NaN & NaN & NaN & NaN & NaN & NaN & NaN & NaN & NaN & NaN & NaN \\
DIRECT PAYMENT - SORGHUM & NaN & NaN & \$ 0.00 & \$ 0.00 & \$ 0.00 & \$ 0.00 & \$ 351.14 & \$ -3.63 & \$ 0.00 & \$ 0.00 & \$ -1.12 & \$ 0.00 & \$ 0.00 & \$ 0.00 & \$ 0.00 & \$ 0.00 & \$ 0.00 & \$ 0.00 & NaN \\
DIRECT PAYMENT - SOYBEANS & NaN & NaN & \$ 0.00 & \$ 0.00 & \$ 0.00 & \$ 0.00 & \$ 258.28 & \$ 28.50 & \$ -9.15 & \$ 0.00 & \$ -2.86 & \$ 0.00 & \$ 0.00 & \$ 0.00 & \$ 0.00 & \$ 0.00 & \$ 0.00 & \$ 0.00 & NaN \\
DIRECT PAYMENT - SUNFLOWER SEED & NaN & NaN & \$ 0.00 & \$ 0.00 & \$ 0.00 & \$ 0.00 & \$ 495.41 & \$ 0.00 & \$ 0.00 & \$ 0.00 & \$ 8.67 & \$ 0.00 & \$ 0.00 & \$ 0.00 & \$ 0.00 & \$ 0.00 & \$ 0.00 & \$ 0.00 & NaN \\
DIRECT PAYMENT - UPLAND COTTON & NaN & NaN & \$ 0.00 & \$ 0.00 & \$ 0.00 & \$ 0.00 & \$ 2,088.15 & \$ 267.26 & \$ 38.18 & \$ 0.00 & \$ 16.39 & \$ 0.00 & \$ 0.00 & \$ 0.00 & \$ 0.00 & \$ 0.00 & \$ 0.00 & \$ 0.00 & NaN \\
DIRECT PAYMENT - WHEAT & NaN & NaN & \$ 0.00 & \$ 0.00 & \$ 0.00 & \$ 0.00 & \$ 889.45 & \$ 2.78 & \$ -201.68 & \$ 0.00 & \$ -150.50 & \$ 0.00 & \$ 0.00 & \$ 0.00 & \$ 0.00 & \$ 0.00 & \$ 0.00 & \$ 0.00 & NaN \\
DIS/WH2 2019 WFHURRINDEMP & NaN & NaN & \$ 0.00 & \$ 0.00 & \$ 0.00 & \$ 0.00 & \$ 0.00 & \$ 0.00 & \$ 0.00 & \$ 0.00 & \$ 0.00 & \$ 0.00 & \$ 0.00 & \$ 0.00 & \$ 0.00 & \$ -54.69 & \$ 34.91 & \$ -8,358.90 & NaN \\
DMC PRGM-SUPPLEMENTAL & NaN & NaN & NaN & NaN & NaN & NaN & NaN & NaN & NaN & NaN & NaN & NaN & NaN & NaN & NaN & NaN & NaN & NaN & NaN \\
DURUM WHEAT QUALITY PROGRAM & NaN & NaN & \$ 0.00 & \$ 0.00 & \$ 0.00 & \$ 0.00 & \$ 15.79 & \$ 0.00 & \$ 0.00 & \$ 0.00 & \$ 0.00 & \$ 0.00 & \$ 0.00 & \$ 0.00 & \$ 0.00 & \$ 0.00 & \$ 0.00 & \$ 0.00 & NaN \\
ECONOMIC ADJUSTMENT ASSIST-UPLAND COTTON & NaN & NaN & NaN & NaN & NaN & NaN & NaN & NaN & NaN & NaN & NaN & NaN & NaN & NaN & NaN & NaN & NaN & NaN & NaN \\
ECP - ADJUSTED GROSS INCOME & NaN & NaN & NaN & NaN & NaN & NaN & NaN & NaN & NaN & NaN & NaN & NaN & NaN & NaN & NaN & NaN & NaN & NaN & NaN \\
ECP COST SHARE FY 2018 & NaN & NaN & \$ 0.00 & \$ 0.00 & \$ 0.00 & \$ 0.00 & \$ 0.00 & \$ 0.00 & \$ 0.00 & \$ 0.00 & \$ 0.00 & \$ 0.00 & \$ 0.00 & \$ 0.00 & \$ -5,362.51 & \$ -8,865.75 & \$ 827.92 & \$ -1,476.19 & NaN \\
ECP HURICNS HARVEY/IRMA/MARIA-CSTSHRFY18 & NaN & NaN & NaN & NaN & NaN & NaN & NaN & NaN & NaN & NaN & NaN & NaN & NaN & NaN & NaN & NaN & NaN & NaN & NaN \\
ECP PRIVATE SECTOR TECHNICAL ASSISTANCE & NaN & NaN & \$ 0.00 & \$ 0.00 & \$ 0.00 & \$ 0.00 & \$ 0.00 & \$ 0.00 & \$ 0.00 & \$ 1,150.00 & \$ 0.00 & \$ 0.00 & \$ 0.00 & \$ 0.00 & \$ 0.00 & \$ 0.00 & \$ 0.00 & \$ 150.00 & NaN \\
ECP SANDY STAFFORD COST SHARE & NaN & NaN & \$ 0.00 & \$ 0.00 & \$ 0.00 & \$ 0.00 & \$ 0.00 & \$ 0.00 & \$ 0.00 & \$ 59.54 & \$ -2.11 & \$ 0.00 & \$ 0.00 & \$ 0.00 & \$ 0.00 & \$ 0.00 & \$ 0.00 & \$ 0.00 & NaN \\
ECP/HMC COST SHARE FY19 & NaN & NaN & \$ 0.00 & \$ 0.00 & \$ 0.00 & \$ 0.00 & \$ 0.00 & \$ 0.00 & \$ 0.00 & \$ 0.00 & \$ 0.00 & \$ 0.00 & \$ 0.00 & \$ 0.00 & \$ 0.00 & \$ -30.66 & \$ 58.00 & \$ 2,014.57 & NaN \\
ECPCOF & NaN & NaN & \$ 0.00 & \$ 0.00 & \$ 0.00 & \$ 0.00 & \$ 0.00 & \$ 0.00 & \$ 0.00 & \$ 0.00 & \$ 0.00 & \$ 0.00 & \$ 0.00 & \$ 0.00 & \$ 0.00 & \$ 0.00 & \$ 264.99 & \$ -2,419.19 & NaN \\
ECPSFCOF & NaN & NaN & \$ 0.00 & \$ 0.00 & \$ 0.00 & \$ 0.00 & \$ 0.00 & \$ 0.00 & \$ 0.00 & \$ 0.00 & \$ 0.00 & \$ 0.00 & \$ 0.00 & \$ 0.00 & \$ 0.00 & \$ 0.00 & \$ 0.00 & \$ 11,875.00 & NaN \\
EFPSTACSOF & NaN & NaN & NaN & NaN & NaN & NaN & NaN & NaN & NaN & NaN & NaN & NaN & NaN & NaN & NaN & NaN & NaN & NaN & NaN \\
EFRP SANDY STAFFORD COST SHARE & NaN & NaN & \$ 0.00 & \$ 0.00 & \$ 0.00 & \$ 0.00 & \$ 0.00 & \$ 0.00 & \$ 0.00 & \$ 0.00 & \$ -168.04 & \$ -22.02 & \$ 364.10 & \$ 5,022.74 & \$ 0.00 & \$ 0.00 & \$ 0.00 & \$ 0.00 & NaN \\
EFRP SANDY STAFFORD COST SHARE                & NaN & NaN & NaN & NaN & NaN & NaN & NaN & NaN & NaN & NaN & NaN & NaN & NaN & NaN & NaN & NaN & NaN & NaN & NaN \\
EFRPCSOF & NaN & NaN & NaN & NaN & NaN & NaN & NaN & NaN & NaN & NaN & NaN & NaN & NaN & NaN & NaN & NaN & NaN & NaN & NaN \\
EMERG ASSIST LIVESTOCK BEES FISH (ELAP) & NaN & NaN & \$ 0.00 & \$ 0.00 & \$ 0.00 & \$ 0.00 & \$ 12,684.15 & \$ 453.33 & \$ -109.71 & \$ -11,434.74 & \$ 268.97 & \$ 219.42 & \$ 1,142.60 & \$ 230.57 & \$ 260.14 & \$ -26.79 & \$ 82.83 & \$ 239.86 & NaN \\
EMERG ASSIST LIVESTOCK BEES FISH (ELAP)       & NaN & NaN & NaN & NaN & NaN & NaN & NaN & NaN & NaN & NaN & NaN & NaN & NaN & NaN & NaN & NaN & NaN & NaN & NaN \\
EMERG CONSERVATION - HURRICANE, AUTO & NaN & NaN & \$ 0.00 & \$ 0.00 & \$ 0.00 & \$ 0.00 & \$ 7,370.18 & \$ 1,448.25 & \$ 0.00 & \$ 0.00 & \$ 0.00 & \$ 0.00 & \$ 0.00 & \$ 0.00 & \$ 0.00 & \$ 0.00 & \$ 0.00 & \$ 0.00 & NaN \\
EMERG CONSV - HURRICANE GULF OF MEXICO & NaN & NaN & \$ 0.00 & \$ 0.00 & \$ 0.00 & \$ 0.00 & \$ 8,357.37 & \$ 3,469.87 & \$ 0.00 & \$ 490.00 & \$ 0.00 & \$ 0.00 & \$ 0.00 & \$ 0.00 & \$ 0.00 & \$ 0.00 & \$ 0.00 & \$ 0.00 & NaN \\
EMERGENCY ASSISTANCE LIVESTOCK, HONEYBEE, FISH & NaN & NaN & NaN & NaN & NaN & NaN & NaN & NaN & NaN & NaN & NaN & NaN & NaN & NaN & NaN & NaN & NaN & NaN & NaN \\
EMERGENCY COMPENSATION PROGRAM - AGI & NaN & NaN & \$ 0.00 & \$ 0.00 & \$ 0.00 & \$ 0.00 & \$ -423.50 & \$ 0.00 & \$ 0.00 & \$ 0.00 & \$ 0.00 & \$ 0.00 & \$ 0.00 & \$ 0.00 & \$ 0.00 & \$ 0.00 & \$ 0.00 & \$ 0.00 & NaN \\
EMERGENCY CONSERV - SOUTHERN CALIFORNIA & NaN & NaN & \$ 0.00 & \$ 0.00 & \$ 0.00 & \$ 0.00 & \$ 15,282.11 & \$ 0.00 & \$ 0.00 & \$ 0.00 & \$ 0.00 & \$ 0.00 & \$ 0.00 & \$ 0.00 & \$ 0.00 & \$ 0.00 & \$ 0.00 & \$ 0.00 & NaN \\
EMERGENCY CONSERVATION - DROUGHT, AUTO & NaN & NaN & \$ 0.00 & \$ 0.00 & \$ 0.00 & \$ 0.00 & \$ 3,248.41 & \$ -71.24 & \$ 2,189.00 & \$ 0.00 & \$ 0.00 & \$ 0.00 & \$ 0.00 & \$ 0.00 & \$ 0.00 & \$ 0.00 & \$ 0.00 & \$ 0.00 & NaN \\
EMERGENCY CONSERVATION - FLOOD & NaN & NaN & \$ 0.00 & \$ 0.00 & \$ 0.00 & \$ 0.00 & \$ 2,267.37 & \$ 4.39 & \$ 39.85 & \$ -48.47 & \$ 0.00 & \$ 0.00 & \$ 0.00 & \$ 0.00 & \$ 0.00 & \$ 0.00 & \$ 0.00 & \$ 0.00 & NaN \\
EMERGENCY CONSERVATION - HURRICANE GULF MEXICO & NaN & NaN & \$ 0.00 & \$ 0.00 & \$ 0.00 & \$ 0.00 & \$ -8,358.98 & \$ 0.00 & \$ 0.00 & \$ 0.00 & \$ 0.00 & \$ 0.00 & \$ 0.00 & \$ 0.00 & \$ 0.00 & \$ 0.00 & \$ 0.00 & \$ 0.00 & NaN \\
EMERGENCY CONSERVATION - SOUTHERN CALIFORNIA & NaN & NaN & \$ 0.00 & \$ 0.00 & \$ 0.00 & \$ 0.00 & \$ -15,282.11 & \$ 0.00 & \$ 0.00 & \$ 0.00 & \$ 0.00 & \$ 0.00 & \$ 0.00 & \$ 0.00 & \$ 0.00 & \$ 0.00 & \$ 0.00 & \$ 0.00 & NaN \\
EMERGENCY CONSERVATION - TECH SERVICES & NaN & NaN & NaN & NaN & NaN & NaN & NaN & NaN & NaN & NaN & NaN & NaN & NaN & NaN & NaN & NaN & NaN & NaN & NaN \\
EMERGENCY CONSERVATION - TORNADO, AUTO & NaN & NaN & \$ 0.00 & \$ 0.00 & \$ 0.00 & \$ 0.00 & \$ 1,725.41 & \$ -5.15 & \$ 0.00 & \$ 0.00 & \$ 0.00 & \$ 0.00 & \$ 0.00 & \$ 0.00 & \$ 0.00 & \$ 0.00 & \$ 0.00 & \$ 0.00 & NaN \\
EMERGENCY CONSERVATION PROGRAM & NaN & NaN & \$ 0.00 & \$ 0.00 & \$ 0.00 & \$ 0.00 & \$ 0.00 & \$ -2.65 & \$ 37.32 & \$ 61.83 & \$ 62.11 & \$ 81.33 & \$ -172.60 & \$ -75.52 & \$ 89.85 & \$ 182.26 & \$ -23.04 & \$ 5,476.67 & NaN \\
EMERGENCY CONSERVATION PROGRAM                & NaN & NaN & NaN & NaN & NaN & NaN & NaN & NaN & NaN & NaN & NaN & NaN & NaN & NaN & NaN & NaN & NaN & NaN & NaN \\
EMERGENCY CONSERVATION PROGRAM - DROUGHT & NaN & NaN & \$ 0.00 & \$ 0.00 & \$ 0.00 & \$ -0.00 & \$ -3,226.65 & \$ 0.00 & \$ 0.00 & \$ 0.00 & \$ 0.00 & \$ 0.00 & \$ 0.00 & \$ 0.00 & \$ 0.00 & \$ 0.00 & \$ 0.00 & \$ 0.00 & NaN \\
EMERGENCY CONSERVATION PROGRAM - FLOOD & NaN & NaN & \$ 0.00 & \$ 0.00 & \$ 0.00 & \$ -0.00 & \$ -2,228.32 & \$ 0.00 & \$ 0.00 & \$ 0.00 & \$ 0.00 & \$ 0.00 & \$ 0.00 & \$ 0.00 & \$ 0.00 & \$ 0.00 & \$ 0.00 & \$ 0.00 & NaN \\
EMERGENCY CONSERVATION PROGRAM - HURRICANE & NaN & NaN & \$ 0.00 & \$ 0.00 & \$ 0.00 & \$ 0.00 & \$ -7,233.71 & \$ 0.00 & \$ 0.00 & \$ 0.00 & \$ 0.00 & \$ 0.00 & \$ 0.00 & \$ 0.00 & \$ 0.00 & \$ 0.00 & \$ 0.00 & \$ 0.00 & NaN \\
EMERGENCY CONSERVATION PROGRAM - KANSAS TORNADOS & NaN & NaN & \$ 0.00 & \$ 0.00 & \$ 0.00 & \$ 0.00 & \$ 0.00 & \$ 0.00 & \$ 0.00 & \$ 0.00 & \$ 0.00 & \$ 0.00 & \$ 0.00 & \$ 0.00 & \$ 0.00 & \$ 0.00 & \$ 0.00 & \$ 0.00 & NaN \\
EMERGENCY CONSERVATION PROGRAM - OTHER & NaN & NaN & \$ 0.00 & \$ 0.00 & \$ 0.00 & \$ 0.00 & \$ -1,344.56 & \$ 0.00 & \$ 0.00 & \$ 0.00 & \$ 0.00 & \$ 0.00 & \$ 0.00 & \$ 0.00 & \$ 0.00 & \$ 0.00 & \$ 0.00 & \$ 0.00 & NaN \\
EMERGENCY CONSERVATION PROGRAM - TORNADO & NaN & NaN & \$ 0.00 & \$ 0.00 & \$ 0.00 & \$ 0.00 & \$ -1,713.71 & \$ 0.00 & \$ 0.00 & \$ 0.00 & \$ 0.00 & \$ 0.00 & \$ 0.00 & \$ 0.00 & \$ 0.00 & \$ 0.00 & \$ 0.00 & \$ 0.00 & NaN \\
EMERGENCY CONSERVATION PROGRAM FY17 & NaN & NaN & \$ 0.00 & \$ 0.00 & \$ 0.00 & \$ 0.00 & \$ 0.00 & \$ 0.00 & \$ 0.00 & \$ 0.00 & \$ 0.00 & \$ 0.00 & \$ 0.00 & \$ 1.64 & \$ 125.65 & \$ 133.61 & \$ 26.62 & \$ 3,042.05 & NaN \\
EMERGENCY CONSERVATION PROGRAM STAFFORD & NaN & NaN & \$ 0.00 & \$ 0.00 & \$ 0.00 & \$ 0.00 & \$ 0.00 & \$ 0.00 & \$ 3.18 & \$ 4.39 & \$ 109.25 & \$ -23.13 & \$ 378.51 & \$ 56.77 & \$ 1,556.94 & \$ -2.63 & \$ -8.13 & \$ 4,771.47 & NaN \\
EMERGENCY CONSERVATION PROGRAM STAFFORD       & NaN & NaN & NaN & NaN & NaN & NaN & NaN & NaN & NaN & NaN & NaN & NaN & NaN & NaN & NaN & NaN & NaN & NaN & NaN \\
EMERGENCY FOREST RESTORATION STAFFORD & NaN & NaN & \$ 0.00 & \$ 0.00 & \$ 0.00 & \$ 0.00 & \$ 0.00 & \$ 0.00 & \$ 133.92 & \$ 133.71 & \$ -91.30 & \$ 287.69 & \$ -488.30 & \$ 76.78 & \$ -7.69 & \$ 0.00 & \$ 0.00 & \$ 8,294.61 & NaN \\
EMERGENCY FOREST RESTORATION STAFFORD         & NaN & NaN & NaN & NaN & NaN & NaN & NaN & NaN & NaN & NaN & NaN & NaN & NaN & NaN & NaN & NaN & NaN & NaN & NaN \\
EMERGENCY FORESTRY RESTORATION PROGRAM & NaN & NaN & \$ 0.00 & \$ 0.00 & \$ 0.00 & \$ 0.00 & \$ 0.00 & \$ 51.41 & \$ 34.08 & \$ 109.13 & \$ 412.35 & \$ -994.72 & \$ -1,383.69 & \$ -162.21 & \$ -129.57 & \$ 14.70 & \$ 0.00 & \$ 11,946.62 & NaN \\
EMERGENCY FORESTRY RESTORATION PROGRAM        & NaN & NaN & NaN & NaN & NaN & NaN & NaN & NaN & NaN & NaN & NaN & NaN & NaN & NaN & NaN & NaN & NaN & NaN & NaN \\
EMERGENCY LIVESTOCK FEED ASSISTANCE & NaN & NaN & \$ 0.00 & \$ 0.00 & \$ 0.00 & \$ 0.00 & \$ 0.00 & \$ 0.00 & \$ 0.00 & \$ 0.00 & \$ 0.00 & \$ 0.00 & \$ 0.00 & \$ 0.00 & \$ 0.00 & \$ 0.00 & \$ 0.00 & \$ 0.00 & NaN \\
EMERGENCY LIVESTOCK RELIEF PROGRAM & NaN & NaN & NaN & NaN & NaN & NaN & NaN & NaN & NaN & NaN & NaN & NaN & NaN & NaN & NaN & NaN & NaN & NaN & NaN \\
EMGNCY RELIEF PRGM-NONSPECIALITY CROPS & NaN & NaN & NaN & NaN & NaN & NaN & NaN & NaN & NaN & NaN & NaN & NaN & NaN & NaN & NaN & NaN & NaN & NaN & NaN \\
EMGNCY RELIEF PROGRAM-SPECIALITY CROPS & NaN & NaN & NaN & NaN & NaN & NaN & NaN & NaN & NaN & NaN & NaN & NaN & NaN & NaN & NaN & NaN & NaN & NaN & NaN \\
ERFMCFSOF & NaN & NaN & \$ 0.00 & \$ 0.00 & \$ 0.00 & \$ 0.00 & \$ 0.00 & \$ 0.00 & \$ 0.00 & \$ 0.00 & \$ 0.00 & \$ 0.00 & \$ 0.00 & \$ 0.00 & \$ 0.00 & \$ 0.00 & \$ 0.00 & \$ 27,848.73 & NaN \\
ERP/HMC COST SHARE FY19 & NaN & NaN & \$ 0.00 & \$ 0.00 & \$ 0.00 & \$ 0.00 & \$ 0.00 & \$ 0.00 & \$ 0.00 & \$ 0.00 & \$ 0.00 & \$ 0.00 & \$ 0.00 & \$ 0.00 & \$ 0.00 & \$ 0.00 & \$ -132.57 & \$ 1,985.69 & NaN \\
EWE LAMB REPLACEMENT OR RETENTION & NaN & NaN & \$ 0.00 & \$ 0.00 & \$ 0.00 & \$ 0.00 & \$ 0.00 & \$ 0.00 & \$ 0.00 & \$ 0.00 & \$ 0.00 & \$ 0.00 & \$ 0.00 & \$ 0.00 & \$ 0.00 & \$ 0.00 & \$ 0.00 & \$ 0.00 & NaN \\
FARM RANCHERS PROGRAM & NaN & NaN & NaN & NaN & NaN & NaN & NaN & NaN & NaN & NaN & NaN & NaN & NaN & NaN & NaN & NaN & NaN & NaN & NaN \\
FEED PROGRAM - AMERICAN INDIAN & NaN & NaN & \$ 0.00 & \$ 0.00 & \$ 0.00 & \$ 0.00 & \$ 0.00 & \$ 0.00 & \$ 0.00 & \$ 0.00 & \$ 0.00 & \$ 0.00 & \$ 0.00 & \$ 0.00 & \$ 0.00 & \$ 0.00 & \$ 0.00 & \$ 0.00 & NaN \\
FLORIDA HURRICANE CHARLEY CITRUS DISASTER & NaN & NaN & \$ 0.00 & \$ 0.00 & \$ 0.00 & \$ 0.00 & \$ 0.00 & \$ 0.00 & \$ 0.00 & \$ 0.00 & \$ 0.00 & \$ 0.00 & \$ 0.00 & \$ 0.00 & \$ 0.00 & \$ 0.00 & \$ 0.00 & \$ 0.00 & NaN \\
FOOD SAFETY CERTFCTN-SPECIALITY CROPS & NaN & NaN & NaN & NaN & NaN & NaN & NaN & NaN & NaN & NaN & NaN & NaN & NaN & NaN & NaN & NaN & NaN & NaN & NaN \\
GEOGRAPHIC DISADVANTAGED PROGRAM & NaN & NaN & \$ 0.00 & \$ 0.00 & \$ 0.00 & \$ 0.00 & \$ 0.00 & \$ -21.03 & \$ -108.70 & \$ -35.71 & \$ -77.99 & \$ -37.05 & \$ -299.91 & \$ -77.15 & \$ -96.01 & \$ -153.43 & \$ -19.39 & \$ 1,475.90 & NaN \\
GEOGRAPHIC DISADVANTAGED PROGRAM              & NaN & NaN & NaN & NaN & NaN & NaN & NaN & NaN & NaN & NaN & NaN & NaN & NaN & NaN & NaN & NaN & NaN & NaN & NaN \\
GRASSLAND RESERVE PROGRAM - EASEMENT & NaN & NaN & \$ 0.00 & \$ 0.00 & \$ 0.00 & \$ 0.00 & \$ 0.00 & \$ 0.00 & \$ 0.00 & \$ 0.00 & \$ 0.00 & \$ 0.00 & \$ 0.00 & \$ 0.00 & \$ 0.00 & \$ 0.00 & \$ 0.00 & \$ 0.00 & NaN \\
GRASSLAND RESERVE PROGRAM-ANNUAL RENTAL & NaN & NaN & \$ 0.00 & \$ 0.00 & \$ 0.00 & \$ 0.00 & \$ 0.00 & \$ 0.00 & \$ 0.00 & \$ 0.00 & \$ 0.00 & \$ 0.00 & \$ -0.05 & \$ 0.00 & \$ 0.00 & \$ 0.00 & \$ 0.00 & \$ 0.00 & NaN \\
GRASSLAND RESERVE PROGRAM-ANNUAL RENTAL       & NaN & NaN & NaN & NaN & NaN & NaN & NaN & NaN & NaN & NaN & NaN & NaN & NaN & NaN & NaN & NaN & NaN & NaN & NaN \\
GRASSLANDS RESERVE PROGRAM & NaN & NaN & \$ 0.00 & \$ 0.00 & \$ 0.00 & \$ 0.00 & \$ 4,428.81 & \$ 2,437.05 & \$ -578.09 & \$ -134.77 & \$ -109.69 & \$ 15.99 & \$ 272.93 & \$ -18.81 & \$ -12.64 & \$ 6.22 & \$ -16.85 & \$ -1,569.74 & NaN \\
GRASSLANDS RESERVE PROGRAM                    & NaN & NaN & NaN & NaN & NaN & NaN & NaN & NaN & NaN & NaN & NaN & NaN & NaN & NaN & NaN & NaN & NaN & NaN & NaN \\
GRASSLANDS RESERVE SPECIAL FUNDING & NaN & NaN & \$ 0.00 & \$ 0.00 & \$ 0.00 & \$ 0.00 & \$ 2,951.62 & \$ 404.39 & \$ 762.30 & \$ 0.00 & \$ 0.00 & \$ 0.00 & \$ 0.00 & \$ 0.00 & \$ 0.00 & \$ 0.00 & \$ 0.00 & \$ 0.00 & NaN \\
GRAZE-OUT - BARLEY & NaN & NaN & NaN & NaN & NaN & NaN & NaN & NaN & NaN & NaN & NaN & NaN & NaN & NaN & NaN & NaN & NaN & NaN & NaN \\
GRAZE-OUT - WHEAT & NaN & NaN & NaN & NaN & NaN & NaN & NaN & NaN & NaN & NaN & NaN & NaN & NaN & NaN & NaN & NaN & NaN & NaN & NaN \\
HAWAII SUGAR DISASTER PROGRAM & NaN & NaN & \$ 0.00 & \$ 0.00 & \$ 0.00 & \$ 0.00 & \$ 4,922,956.76 & \$ 0.00 & \$ 0.00 & \$ 0.00 & \$ 0.00 & \$ 0.00 & \$ 0.00 & \$ 0.00 & \$ 0.00 & \$ 0.00 & \$ 0.00 & \$ 0.00 & NaN \\
HAY GRAZING & NaN & NaN & \$ 0.00 & \$ 0.00 & \$ 0.00 & \$ 0.00 & \$ 390.76 & \$ -33.86 & \$ 8.38 & \$ 12.07 & \$ 0.03 & \$ -0.05 & \$ -0.03 & \$ 0.00 & \$ 0.00 & \$ 0.00 & \$ 0.00 & \$ 0.00 & NaN \\
HAY GRAZING, MANAGED & NaN & NaN & \$ 0.00 & \$ 0.00 & \$ 0.00 & \$ 0.00 & \$ 391.95 & \$ -2.45 & \$ 9.03 & \$ 5.65 & \$ 13.13 & \$ 5.28 & \$ 4.17 & \$ 0.00 & \$ 0.00 & \$ 0.00 & \$ 0.00 & \$ 0.00 & NaN \\
HURRICANE INDEMNITY PROGRAM & NaN & NaN & \$ 0.00 & \$ 0.00 & \$ 0.00 & \$ 0.00 & \$ 0.00 & \$ 0.00 & \$ 0.00 & \$ 0.00 & \$ 0.00 & \$ 0.00 & \$ 0.00 & \$ 0.00 & \$ 0.00 & \$ 0.00 & \$ 0.00 & \$ 0.00 & NaN \\
HURRICANE INDEMNITY PROGRAM, AUTHORIZED & NaN & NaN & \$ 0.00 & \$ 0.00 & \$ 0.00 & \$ 0.07 & \$ -212.08 & \$ -10,076.00 & \$ 0.00 & \$ 0.00 & \$ 0.00 & \$ 0.00 & \$ 0.00 & \$ 0.00 & \$ 0.00 & \$ 0.00 & \$ 0.00 & \$ 0.00 & NaN \\
INCENTIVE PAYMENTS - RIPARIAN BUFFER & NaN & NaN & \$ 0.00 & \$ 0.00 & \$ 0.00 & \$ 0.00 & \$ 6,894.03 & \$ 0.00 & \$ 0.00 & \$ 0.00 & \$ 0.00 & \$ 0.00 & \$ 0.00 & \$ 0.00 & \$ 0.00 & \$ 0.00 & \$ 0.00 & \$ 0.00 & NaN \\
INCOME LOSS - MILK & NaN & NaN & \$ 0.00 & \$ 0.00 & \$ 0.00 & \$ 0.00 & \$ 0.00 & \$ 0.00 & \$ 0.00 & \$ 0.00 & \$ 0.00 & \$ 0.00 & \$ 0.00 & \$ 0.00 & \$ 0.00 & \$ 0.00 & \$ 0.00 & \$ 0.00 & NaN \\
INCOME LOSS - MILK, PART 2 & NaN & NaN & \$ 0.00 & \$ 0.00 & \$ 0.00 & \$ 0.00 & \$ 570.53 & \$ 561.73 & \$ 13.09 & \$ 11.28 & \$ 221.28 & \$ 110.80 & \$ 0.24 & \$ -0.06 & \$ 0.00 & \$ 0.00 & \$ 0.00 & \$ 0.00 & NaN \\
INCOME LOSS TRANSITION - MILK & NaN & NaN & \$ 0.00 & \$ 0.00 & \$ 0.00 & \$ 0.00 & \$ 0.00 & \$ 0.00 & \$ 0.00 & \$ 0.00 & \$ 0.00 & \$ 0.00 & \$ 0.00 & \$ 0.00 & \$ 0.00 & \$ 0.00 & \$ 0.00 & \$ 0.00 & NaN \\
INDEMNITY PAYMENT - DAIRY & NaN & NaN & \$ 0.00 & \$ 0.00 & \$ 0.00 & \$ 0.00 & \$ 10,665.42 & \$ -90.72 & \$ -0.00 & \$ 26.06 & \$ 10.82 & \$ -0.02 & \$ -8,081.58 & \$ 8.04 & \$ 0.00 & \$ -0.00 & \$ -0.00 & \$ 0.00 & NaN \\
INDEMNITY PAYMENT - DAIRY                     & NaN & NaN & NaN & NaN & NaN & NaN & NaN & NaN & NaN & NaN & NaN & NaN & NaN & NaN & NaN & NaN & NaN & NaN & NaN \\
INTEREST PENALTY & NaN & NaN & \$ 0.00 & \$ 0.00 & \$ 0.01 & \$ 0.01 & \$ -45.78 & \$ 0.00 & \$ 0.00 & \$ 0.00 & \$ 0.00 & \$ 0.00 & \$ 0.00 & \$ 0.00 & \$ 0.00 & \$ 0.00 & \$ 0.00 & \$ 0.00 & NaN \\
LIVESTOCK ASSISTANCE PROGRAM & NaN & NaN & \$ 0.00 & \$ 0.00 & \$ 0.00 & \$ 0.00 & \$ 0.00 & \$ 0.00 & \$ 0.00 & \$ 0.00 & \$ 0.00 & \$ 0.00 & \$ 0.00 & \$ 0.00 & \$ 0.00 & \$ 0.00 & \$ 0.00 & \$ 0.00 & NaN \\
LIVESTOCK COMPENSATION PROGRAM & NaN & NaN & \$ 0.00 & \$ 0.00 & \$ 0.00 & \$ 0.00 & \$ 0.00 & \$ 0.00 & \$ 0.00 & \$ 0.00 & \$ 0.00 & \$ 0.00 & \$ 0.00 & \$ 0.00 & \$ 0.00 & \$ 0.00 & \$ 0.00 & \$ 0.00 & NaN \\
LIVESTOCK COMPENSATION PROGRAM - HURRICANE & NaN & NaN & \$ 0.00 & \$ 0.00 & \$ 0.00 & \$ 0.00 & \$ 0.00 & \$ 0.00 & \$ 0.00 & \$ 0.00 & \$ 0.00 & \$ 0.00 & \$ 0.00 & \$ 0.00 & \$ 0.00 & \$ 0.00 & \$ 0.00 & \$ 0.00 & NaN \\
LIVESTOCK FORAGE DISASTER  PROGRAM & NaN & NaN & \$ 0.00 & \$ 0.00 & \$ 0.00 & \$ 0.00 & \$ -4,895.71 & \$ 0.00 & \$ 0.00 & \$ 0.00 & \$ 0.00 & \$ 0.00 & \$ 0.00 & \$ 0.00 & \$ 0.00 & \$ 0.00 & \$ 0.00 & \$ 0.00 & NaN \\
LIVESTOCK FORAGE DISASTER PROGRAM & NaN & NaN & \$ 0.00 & \$ 0.00 & \$ 0.00 & \$ 0.00 & \$ 4,892.13 & \$ 63.61 & \$ 200.54 & \$ 3,027.02 & \$ 2,533.78 & \$ -2,216.53 & \$ -1,912.26 & \$ 3,397.30 & \$ 3,959.46 & \$ 5,001.18 & \$ 1,312.00 & \$ 2,864.75 & NaN \\
LIVESTOCK FORAGE PROGRAM & NaN & NaN & \$ 0.00 & \$ 0.00 & \$ 0.00 & \$ 0.00 & \$ 0.00 & \$ 0.00 & \$ 0.00 & \$ 0.00 & \$ 157.83 & \$ 167.67 & \$ -33.87 & \$ 105.92 & \$ 145.25 & \$ 88.27 & \$ 11.27 & \$ 512.08 & NaN \\
LIVESTOCK FORAGE PROGRAM                      & NaN & NaN & NaN & NaN & NaN & NaN & NaN & NaN & NaN & NaN & NaN & NaN & NaN & NaN & NaN & NaN & NaN & NaN & NaN \\
LIVESTOCK INDEMINITY PAYMENTS PROGRAM & NaN & NaN & NaN & NaN & NaN & NaN & NaN & NaN & NaN & NaN & NaN & NaN & NaN & NaN & NaN & NaN & NaN & NaN & NaN \\
LIVESTOCK INDEMNITY - TRUST FUND & NaN & NaN & \$ 0.00 & \$ 0.00 & \$ 0.00 & \$ 0.00 & \$ 6,705.23 & \$ 0.00 & \$ 0.00 & \$ 0.00 & \$ 0.00 & \$ 0.00 & \$ 0.00 & \$ 0.00 & \$ 0.00 & \$ 0.00 & \$ 0.00 & \$ 0.00 & NaN \\
LIVESTOCK INDEMNITY PAYMENTS & NaN & NaN & NaN & NaN & NaN & NaN & NaN & NaN & NaN & NaN & NaN & NaN & NaN & NaN & NaN & NaN & NaN & NaN & NaN \\
LIVESTOCK INDEMNITY PROGRAM & NaN & NaN & \$ 0.00 & \$ 0.00 & \$ 0.00 & \$ -0.00 & \$ 1,187.39 & \$ 149.39 & \$ -12.05 & \$ 2,424.96 & \$ -2,316.17 & \$ -1,566.87 & \$ 1,585.19 & \$ -2,446.74 & \$ 209.01 & \$ -6,376.86 & \$ 1,128.04 & \$ -224.75 & NaN \\
LIVESTOCK INDEMNITY PROGRAM                   & NaN & NaN & NaN & NaN & NaN & NaN & NaN & NaN & NaN & NaN & NaN & NaN & NaN & NaN & NaN & NaN & NaN & NaN & NaN \\
LIVESTOCK INDEMNITY PROGRAM AUTHORIZED & NaN & NaN & NaN & NaN & NaN & NaN & NaN & NaN & NaN & NaN & NaN & NaN & NaN & NaN & NaN & NaN & NaN & NaN & NaN \\
LOAN DEFICIENCY & NaN & NaN & \$ 0.00 & \$ 0.00 & \$ 0.00 & \$ -0.00 & \$ -4,082.97 & \$ 0.00 & \$ 0.00 & \$ 0.00 & \$ 0.00 & \$ 0.00 & \$ 0.00 & \$ 0.00 & \$ 0.00 & \$ 0.00 & \$ 0.00 & \$ 0.00 & NaN \\
LOSS ADJUSTER - NAP & NaN & NaN & \$ 0.00 & \$ 0.00 & \$ 0.00 & \$ 0.00 & \$ 335.69 & \$ 5.27 & \$ 1.69 & \$ 0.00 & \$ 0.00 & \$ 0.00 & \$ 0.00 & \$ 0.00 & \$ 0.00 & \$ 0.00 & \$ 0.00 & \$ 0.00 & NaN \\
MARGIN PROTECTION  - DAIRY & NaN & NaN & \$ 0.00 & \$ 0.00 & \$ 0.00 & \$ 0.00 & \$ 0.00 & \$ 0.00 & \$ 0.00 & \$ 0.00 & \$ 0.00 & \$ 0.00 & \$ 0.00 & \$ 0.00 & \$ 0.00 & \$ 21.22 & \$ 0.00 & \$ 6,632.00 & NaN \\
MARGIN PROTECTION PROGRAM - DAIRY & NaN & NaN & \$ 0.00 & \$ 0.00 & \$ 0.00 & \$ 0.00 & \$ 0.00 & \$ 0.00 & \$ 0.00 & \$ 0.00 & \$ 0.00 & \$ -116.28 & \$ 164.37 & \$ 436.17 & \$ 270.21 & \$ 36.53 & \$ 103.71 & \$ 395.82 & NaN \\
MARGIN PROTECTION PROGRAM - DAIRY             & NaN & NaN & NaN & NaN & NaN & NaN & NaN & NaN & NaN & NaN & NaN & NaN & NaN & NaN & NaN & NaN & NaN & NaN & NaN \\
MARKET ACCESS PROGRAM & NaN & NaN & NaN & NaN & NaN & NaN & NaN & NaN & NaN & NaN & NaN & NaN & NaN & NaN & NaN & NaN & NaN & NaN & NaN \\
MARKET FACILITATION PROG-SPECIALTY CROPS & NaN & NaN & \$ 0.00 & \$ 0.00 & \$ 0.00 & \$ 0.00 & \$ 0.00 & \$ 0.00 & \$ 0.00 & \$ 0.00 & \$ 0.00 & \$ 0.00 & \$ 0.00 & \$ 0.00 & \$ 72.73 & \$ 188.02 & \$ 219.94 & \$ -13,623.60 & NaN \\
MARKET FACILITATION PROGRAM - CROPS & NaN & NaN & \$ 0.00 & \$ 0.00 & \$ 0.00 & \$ 0.00 & \$ 0.00 & \$ 0.00 & \$ 0.00 & \$ 0.00 & \$ 0.00 & \$ 0.00 & \$ 0.00 & \$ 0.00 & \$ -241.36 & \$ -434.91 & \$ 290.52 & \$ -315.26 & NaN \\
MARKET FACILITATION PROGRAM - DAHG & NaN & NaN & \$ 0.00 & \$ 0.00 & \$ 0.00 & \$ 0.00 & \$ 0.00 & \$ 0.00 & \$ 0.00 & \$ 0.00 & \$ 0.00 & \$ 0.00 & \$ 0.00 & \$ 0.00 & \$ 55.02 & \$ 69.91 & \$ -708.73 & \$ 7,125.50 & NaN \\
MARKET GAINS & NaN & NaN & \$ 0.00 & \$ 0.00 & \$ 0.00 & \$ 0.00 & \$ -7,126.16 & \$ 0.00 & \$ 0.00 & \$ 0.00 & \$ 0.00 & \$ 0.00 & \$ 0.00 & \$ 0.00 & \$ 0.00 & \$ 0.00 & \$ 0.00 & \$ 0.00 & NaN \\
MARKET LOSS ASSISTANCE & NaN & NaN & \$ 0.00 & \$ 0.00 & \$ 0.00 & \$ 0.00 & \$ 0.00 & \$ 0.00 & \$ 0.00 & \$ 0.00 & \$ 0.00 & \$ 0.00 & \$ 0.00 & \$ 0.00 & \$ 0.00 & \$ 0.00 & \$ 0.00 & \$ 0.00 & NaN \\
MARKET LOSS ASSISTANCE-FRESH ASPARAGUS & NaN & NaN & NaN & NaN & NaN & NaN & NaN & NaN & NaN & NaN & NaN & NaN & NaN & NaN & NaN & NaN & NaN & NaN & NaN \\
MARKET LOSS ASSISTANCE-PROCESS ASPARAGUS & NaN & NaN & NaN & NaN & NaN & NaN & NaN & NaN & NaN & NaN & NaN & NaN & NaN & NaN & NaN & NaN & NaN & NaN & NaN \\
MILK INCOME LOSS CONTRACT II & NaN & NaN & \$ 0.00 & \$ 0.00 & \$ 0.01 & \$ -0.00 & \$ -553.28 & \$ 0.00 & \$ 0.00 & \$ 0.00 & \$ 0.00 & \$ 0.00 & \$ 0.00 & \$ 0.00 & \$ 0.00 & \$ 0.00 & \$ 0.00 & \$ 0.00 & NaN \\
NAP REGULAR WEB-BASED & NaN & NaN & \$ 0.00 & \$ 0.00 & \$ 0.00 & \$ 0.00 & \$ 0.00 & \$ 0.00 & \$ 0.00 & \$ 0.00 & \$ 0.00 & \$ 0.00 & \$ 0.00 & \$ 0.00 & \$ 0.00 & \$ 3,995.86 & \$ 2,603.53 & \$ 2,491.83 & NaN \\
NON-INSURED ASSISTANCE FROST FREE & NaN & NaN & \$ 0.00 & \$ 0.00 & \$ 0.00 & \$ 0.00 & \$ 0.00 & \$ 0.00 & \$ 0.00 & \$ 0.00 & \$ 2,304.94 & \$ 1,098.83 & \$ 10,774.95 & \$ 0.00 & \$ 0.00 & \$ 0.00 & \$ 0.00 & \$ 0.00 & NaN \\
NON-INSURED ASSISTANCE FROST FREE             & NaN & NaN & NaN & NaN & NaN & NaN & NaN & NaN & NaN & NaN & NaN & NaN & NaN & NaN & NaN & NaN & NaN & NaN & NaN \\
NON-INSURED ASSISTANCE PROG AUTHORIZED & NaN & NaN & \$ 0.00 & \$ 0.00 & \$ 0.00 & \$ 0.00 & \$ 6,549.95 & \$ 186.49 & \$ -595.39 & \$ 1,889.74 & \$ 3,154.48 & \$ -766.52 & \$ -5,778.20 & \$ 1,512.27 & \$ 5,039.33 & \$ -51,728.78 & \$ 20,324.74 & \$ 0.00 & NaN \\
NON-INSURED ASSISTANCE PROG AUTHORIZED        & NaN & NaN & NaN & NaN & NaN & NaN & NaN & NaN & NaN & NaN & NaN & NaN & NaN & NaN & NaN & NaN & NaN & NaN & NaN \\
NON-INSURED ASSISTANCE PROGRAM & NaN & NaN & \$ 0.00 & \$ 0.00 & \$ 0.00 & \$ 0.00 & \$ 0.00 & \$ 34.06 & \$ 161.16 & \$ 148.07 & \$ 172.89 & \$ 139.89 & \$ 3,266.86 & \$ 319.70 & \$ -38.82 & \$ 10,034.77 & \$ 453.93 & \$ 493.98 & NaN \\
NON-INSURED ASSISTANCE PROGRAM                & NaN & NaN & NaN & NaN & NaN & NaN & NaN & NaN & NaN & NaN & NaN & NaN & NaN & NaN & NaN & NaN & NaN & NaN & NaN \\
NONINSURED ASSISTANCE PROGRAM & NaN & NaN & \$ 0.00 & \$ 0.00 & \$ -0.00 & \$ -0.00 & \$ 48.63 & \$ 18.66 & \$ 307.55 & \$ 0.00 & \$ 0.00 & \$ 0.00 & \$ 0.00 & \$ 0.00 & \$ 0.00 & \$ 0.00 & \$ 0.00 & \$ 0.00 & NaN \\
ORGANIC & TRANSITIONAL EDU & CERT PRGM & NaN & NaN & NaN & NaN & NaN & NaN & NaN & NaN & NaN & NaN & NaN & NaN & NaN & NaN & NaN & NaN & NaN & NaN & NaN \\
ORGANIC COST SHARE FEES- ST ORG PGM FEES & NaN & NaN & \$ 0.00 & \$ 0.00 & \$ 0.00 & \$ 0.00 & \$ 0.00 & \$ 0.00 & \$ 0.00 & \$ 0.00 & \$ 0.00 & \$ 0.00 & \$ 0.00 & \$ 0.00 & \$ 0.00 & \$ 0.00 & \$ 0.00 & \$ 270.48 & NaN \\
ORIENTAL FRUIT FLY & NaN & NaN & NaN & NaN & NaN & NaN & NaN & NaN & NaN & NaN & NaN & NaN & NaN & NaN & NaN & NaN & NaN & NaN & NaN \\
PANDEMIC ASST-TIMBER HARVESTERS/HAULERS & NaN & NaN & NaN & NaN & NaN & NaN & NaN & NaN & NaN & NaN & NaN & NaN & NaN & NaN & NaN & NaN & NaN & NaN & NaN \\
PANDEMIC LIVESTOCK INDEMNITY PROGRAM & NaN & NaN & NaN & NaN & NaN & NaN & NaN & NaN & NaN & NaN & NaN & NaN & NaN & NaN & NaN & NaN & NaN & NaN & NaN \\
PECAN TREES - ADDTNL SUPLMNTL APPROP & NaN & NaN & NaN & NaN & NaN & NaN & NaN & NaN & NaN & NaN & NaN & NaN & NaN & NaN & NaN & NaN & NaN & NaN & NaN \\
PRICE LOSS COVERAGE & NaN & NaN & NaN & NaN & NaN & NaN & NaN & NaN & NaN & NaN & NaN & NaN & NaN & NaN & NaN & NaN & NaN & NaN & NaN \\
PRICE LOSS COVERAGE PROGRAM & NaN & NaN & \$ 0.00 & \$ 0.00 & \$ 0.00 & \$ 0.00 & \$ 0.00 & \$ 0.00 & \$ 0.00 & \$ 0.00 & \$ 0.00 & \$ -5,574.62 & \$ -2,276.89 & \$ -4,698.82 & \$ -2,907.66 & \$ -2,193.08 & \$ 324.91 & \$ 250.33 & NaN \\
PRICE LOSS COVERAGE PROGRAM                   & NaN & NaN & NaN & NaN & NaN & NaN & NaN & NaN & NaN & NaN & NaN & NaN & NaN & NaN & NaN & NaN & NaN & NaN & NaN \\
QUALITY LOSS ADJUSTMENT PROGRAM & NaN & NaN & NaN & NaN & NaN & NaN & NaN & NaN & NaN & NaN & NaN & NaN & NaN & NaN & NaN & NaN & NaN & NaN & NaN \\
SEAFOOD TRADE RELIEF PROGRAM & NaN & NaN & \$ 0.00 & \$ 0.00 & \$ 0.00 & \$ 0.00 & \$ 0.00 & \$ 0.00 & \$ 0.00 & \$ 0.00 & \$ 0.00 & \$ 0.00 & \$ 0.00 & \$ 0.00 & \$ 0.00 & \$ 0.00 & \$ 93.76 & \$ -2,657.64 & NaN \\
SPECIALITY CROP HURRICANE DISASTER - CITRUS & NaN & NaN & \$ 0.00 & \$ 0.00 & \$ 0.00 & \$ 0.00 & \$ 0.00 & \$ 0.00 & \$ 0.00 & \$ 0.00 & \$ 0.00 & \$ 0.00 & \$ 0.00 & \$ 0.00 & \$ 0.00 & \$ 0.00 & \$ 0.00 & \$ 0.00 & NaN \\
SPECIALITY CROP HURRICANE DISASTER - FRUIT/VEG & NaN & NaN & \$ 0.00 & \$ 0.00 & \$ 0.00 & \$ 0.00 & \$ 0.00 & \$ 0.00 & \$ 0.00 & \$ 0.00 & \$ 0.00 & \$ 0.00 & \$ 0.00 & \$ 0.00 & \$ 0.00 & \$ 0.00 & \$ 0.00 & \$ 0.00 & NaN \\
SPECIALITY CROP HURRICANE DISASTER - NURSERY & NaN & NaN & \$ 0.00 & \$ 0.00 & \$ 0.00 & \$ 0.00 & \$ 0.00 & \$ 0.00 & \$ 0.00 & \$ 0.00 & \$ 0.00 & \$ 0.00 & \$ 0.00 & \$ 0.00 & \$ 0.00 & \$ 0.00 & \$ 0.00 & \$ 0.00 & NaN \\
SPOT MARKET HOG PANDEMIC PROGRAM & NaN & NaN & NaN & NaN & NaN & NaN & NaN & NaN & NaN & NaN & NaN & NaN & NaN & NaN & NaN & NaN & NaN & NaN & NaN \\
STORAGE FORGIVEN & NaN & NaN & \$ 0.00 & \$ 0.00 & \$ 0.00 & \$ -0.00 & \$ 0.00 & \$ 0.00 & \$ 0.00 & \$ 0.00 & \$ 0.00 & \$ 0.00 & \$ 0.00 & \$ 0.00 & \$ 0.00 & \$ 0.00 & \$ 0.00 & \$ 0.00 & NaN \\
SUPL REVENUE ASSISTANCE - RECOVERY ACT & NaN & NaN & \$ 0.00 & \$ 0.00 & \$ 0.00 & \$ 0.00 & \$ 7,703.34 & \$ 1,091.10 & \$ 416.96 & \$ 13,141.96 & \$ 0.00 & \$ 5,265.00 & \$ 0.00 & \$ 0.00 & \$ 0.00 & \$ 0.00 & \$ 0.00 & \$ 0.00 & NaN \\
SUPL REVENUE ASSISTANCE RECOVERY ACT & NaN & NaN & NaN & NaN & NaN & NaN & NaN & NaN & NaN & NaN & NaN & NaN & NaN & NaN & NaN & NaN & NaN & NaN & NaN \\
SUPPLEMENTAL REVENUE ASSISTANCE PROGRAM & NaN & NaN & \$ 0.00 & \$ 0.00 & \$ 0.00 & \$ 0.00 & \$ -167.62 & \$ 90.69 & \$ -757.97 & \$ -1,136.95 & \$ 429.66 & \$ 7,940.78 & \$ 2,685.60 & \$ 384.79 & \$ -13,364.00 & \$ 0.00 & \$ 0.00 & \$ 0.00 & NaN \\
SUPPLEMENTAL REVENUE ASSISTANCE PROGRAM       & NaN & NaN & NaN & NaN & NaN & NaN & NaN & NaN & NaN & NaN & NaN & NaN & NaN & NaN & NaN & NaN & NaN & NaN & NaN \\
TAAF FOR 2010 WEB BASED APPLICATION & NaN & NaN & NaN & NaN & NaN & NaN & NaN & NaN & NaN & NaN & NaN & NaN & NaN & NaN & NaN & NaN & NaN & NaN & NaN \\
TAAF FOR 2011 WEB-BASED APPLICATION & NaN & NaN & NaN & NaN & NaN & NaN & NaN & NaN & NaN & NaN & NaN & NaN & NaN & NaN & NaN & NaN & NaN & NaN & NaN \\
TASCP & NaN & NaN & NaN & NaN & NaN & NaN & NaN & NaN & NaN & NaN & NaN & NaN & NaN & NaN & NaN & NaN & NaN & NaN & NaN \\
TMP/MFP 2019 LIVESTOCK & NaN & NaN & \$ 0.00 & \$ 0.00 & \$ 0.00 & \$ 0.00 & \$ 0.00 & \$ 0.00 & \$ 0.00 & \$ 0.00 & \$ 0.00 & \$ 0.00 & \$ 0.00 & \$ 0.00 & \$ 0.00 & \$ 88.08 & \$ 62.24 & \$ 26,968.36 & NaN \\
TMP/MFP 2019 NON SPECIALTY CROPS & NaN & NaN & \$ 0.00 & \$ 0.00 & \$ 0.00 & \$ 0.00 & \$ 0.00 & \$ 0.00 & \$ 0.00 & \$ 0.00 & \$ 0.00 & \$ 0.00 & \$ 0.00 & \$ 0.00 & \$ 0.00 & \$ -328.75 & \$ -169.91 & \$ -130.09 & NaN \\
TMP/MFP 2019 NON-SPECIALITY CROPS-A & NaN & NaN & \$ 0.00 & \$ 0.00 & \$ 0.00 & \$ 0.00 & \$ 0.00 & \$ 0.00 & \$ 0.00 & \$ 0.00 & \$ 0.00 & \$ 0.00 & \$ 0.00 & \$ 0.00 & \$ 0.00 & \$ -3.75 & \$ 21.16 & \$ 2,668.80 & NaN \\
TMP/MFP 2019 SPECIALITY CROPS & NaN & NaN & \$ 0.00 & \$ 0.00 & \$ 0.00 & \$ 0.00 & \$ 0.00 & \$ 0.00 & \$ 0.00 & \$ 0.00 & \$ 0.00 & \$ 0.00 & \$ 0.00 & \$ 0.00 & \$ 0.00 & \$ 93.34 & \$ 169.81 & \$ 605.72 & NaN \\
TOBACCO QUOTA HOLDER - INTEREST & NaN & NaN & \$ 0.00 & \$ 0.00 & \$ 0.00 & \$ 0.00 & \$ -19.96 & \$ 0.00 & \$ 0.00 & \$ 0.00 & \$ 0.00 & \$ 0.00 & \$ 0.00 & \$ 0.00 & \$ 0.00 & \$ 0.00 & \$ 0.00 & \$ 0.00 & NaN \\
TOBACCO TRANSITION PYMNT-AIR CURED, PROD & NaN & NaN & NaN & NaN & NaN & NaN & NaN & NaN & NaN & NaN & NaN & NaN & NaN & NaN & NaN & NaN & NaN & NaN & NaN \\
TOBACCO TRANSITION PYMNT-BURLEY, QUOTA & NaN & NaN & NaN & NaN & NaN & NaN & NaN & NaN & NaN & NaN & NaN & NaN & NaN & NaN & NaN & NaN & NaN & NaN & NaN \\
TOBACCO TRANSITION PYMNT-CIGAR, PROD & NaN & NaN & NaN & NaN & NaN & NaN & NaN & NaN & NaN & NaN & NaN & NaN & NaN & NaN & NaN & NaN & NaN & NaN & NaN \\
TOBACCO TRANSITION PYMNT-FLUE CURED,PROD & NaN & NaN & NaN & NaN & NaN & NaN & NaN & NaN & NaN & NaN & NaN & NaN & NaN & NaN & NaN & NaN & NaN & NaN & NaN \\
TOBACCO TRANSITION PYMNT-FLUECURED,QUOTA & NaN & NaN & NaN & NaN & NaN & NaN & NaN & NaN & NaN & NaN & NaN & NaN & NaN & NaN & NaN & NaN & NaN & NaN & NaN \\
TOBACCO TRANSITION PYMNT-SUN CURED, PROD & NaN & NaN & NaN & NaN & NaN & NaN & NaN & NaN & NaN & NaN & NaN & NaN & NaN & NaN & NaN & NaN & NaN & NaN & NaN \\
TOBACCO TRANSITION PYMNT-SUN CURED,QUOTA & NaN & NaN & NaN & NaN & NaN & NaN & NaN & NaN & NaN & NaN & NaN & NaN & NaN & NaN & NaN & NaN & NaN & NaN & NaN \\
TOBACCO TRANSITION PYMNT-VA FIRE CURED & NaN & NaN & NaN & NaN & NaN & NaN & NaN & NaN & NaN & NaN & NaN & NaN & NaN & NaN & NaN & NaN & NaN & NaN & NaN \\
TOBACCO TRANSITION PYMT-AIR CURED, QUOTA & NaN & NaN & NaN & NaN & NaN & NaN & NaN & NaN & NaN & NaN & NaN & NaN & NaN & NaN & NaN & NaN & NaN & NaN & NaN \\
TOBACCO TRANSITION PYMT-BURLEY, PROD & NaN & NaN & \$ 0.00 & \$ 0.00 & \$ 0.00 & \$ 0.00 & \$ 0.00 & \$ 0.00 & \$ 0.00 & \$ 0.00 & \$ 0.00 & \$ 0.00 & \$ 0.00 & \$ 0.00 & \$ 0.00 & \$ 0.00 & \$ 0.00 & \$ 0.00 & NaN \\
TOBACCO TRANSITION PYMT-CIGAR, QUOTA & NaN & NaN & NaN & NaN & NaN & NaN & NaN & NaN & NaN & NaN & NaN & NaN & NaN & NaN & NaN & NaN & NaN & NaN & NaN \\
TOBACCO TRANSITION PYMT-FIRE CURED, PROD & NaN & NaN & NaN & NaN & NaN & NaN & NaN & NaN & NaN & NaN & NaN & NaN & NaN & NaN & NaN & NaN & NaN & NaN & NaN \\
TOBACCO TRANSITION PYMT-FIRE CURED,QUOTA & NaN & NaN & NaN & NaN & NaN & NaN & NaN & NaN & NaN & NaN & NaN & NaN & NaN & NaN & NaN & NaN & NaN & NaN & NaN \\
TRADE ADJUSTMENT - AVOCADOS, FRESH & NaN & NaN & \$ 0.00 & \$ 0.00 & \$ 0.00 & \$ 0.00 & \$ 0.00 & \$ 0.00 & \$ 0.00 & \$ 0.00 & \$ 0.00 & \$ 0.00 & \$ 0.00 & \$ 0.00 & \$ 0.00 & \$ 0.00 & \$ 0.00 & \$ 0.00 & NaN \\
TRADE ADJUSTMENT - FISH, SALMON & NaN & NaN & \$ 0.00 & \$ 0.00 & \$ 0.00 & \$ 0.00 & \$ 0.00 & \$ 0.00 & \$ 0.00 & \$ 0.00 & \$ 0.00 & \$ 0.00 & \$ 0.00 & \$ 0.00 & \$ 0.00 & \$ 0.00 & \$ 0.00 & \$ 0.00 & NaN \\
TRADE ADJUSTMENT - FISH, SHRIMP & NaN & NaN & \$ 0.00 & \$ 0.00 & \$ 0.00 & \$ 0.00 & \$ 0.00 & \$ 0.00 & \$ 0.00 & \$ 0.00 & \$ 0.00 & \$ 0.00 & \$ 0.00 & \$ 0.00 & \$ 0.00 & \$ 0.00 & \$ 0.00 & \$ 0.00 & NaN \\
TRADE ADJUSTMENT - FRESH POTATOES & NaN & NaN & \$ 0.00 & \$ 0.00 & \$ 0.00 & \$ 0.00 & \$ 0.00 & \$ 0.00 & \$ 0.00 & \$ 0.00 & \$ 0.00 & \$ 0.00 & \$ 0.00 & \$ 0.00 & \$ 0.00 & \$ 0.00 & \$ 0.00 & \$ 0.00 & NaN \\
TRADE ADJUSTMENT - JUICE GRAPES & NaN & NaN & \$ 0.00 & \$ 0.00 & \$ 0.00 & \$ 0.00 & \$ 0.00 & \$ 0.00 & \$ 0.00 & \$ 0.00 & \$ 0.00 & \$ 0.00 & \$ 0.00 & \$ 0.00 & \$ 0.00 & \$ 0.00 & \$ 0.00 & \$ 0.00 & NaN \\
TRADE ADJUSTMENT - NUTS, LYCHEES & NaN & NaN & \$ 0.00 & \$ 0.00 & \$ 0.00 & \$ 0.00 & \$ 0.00 & \$ 0.00 & \$ 0.00 & \$ 0.00 & \$ 0.00 & \$ 0.00 & \$ 0.00 & \$ 0.00 & \$ 0.00 & \$ 0.00 & \$ 0.00 & \$ 0.00 & NaN \\
TRADE ADJUSTMENT - OLIVES & NaN & NaN & \$ 0.00 & \$ 0.00 & \$ 0.00 & \$ 0.00 & \$ 0.00 & \$ 0.00 & \$ 0.00 & \$ 0.00 & \$ 0.00 & \$ 0.00 & \$ 0.00 & \$ 0.00 & \$ 0.00 & \$ 0.00 & \$ 0.00 & \$ 0.00 & NaN \\
TRADE ADJUSTMENT ASSISTANCE & NaN & NaN & \$ 0.00 & \$ 0.00 & \$ -0.02 & \$ -0.15 & \$ 0.00 & \$ 0.00 & \$ 0.00 & \$ 0.00 & \$ 0.00 & \$ 0.00 & \$ 0.00 & \$ 0.00 & \$ 0.00 & \$ 0.00 & \$ 0.00 & \$ 0.00 & NaN \\
TRADE ADJUSTMENT ASSISTANCE FOR FARMERS & NaN & NaN & NaN & NaN & NaN & NaN & NaN & NaN & NaN & NaN & NaN & NaN & NaN & NaN & NaN & NaN & NaN & NaN & NaN \\
TRADE ADJUSTMENT ASSISTANCE FOR FARMERS       & NaN & NaN & NaN & NaN & NaN & NaN & NaN & NaN & NaN & NaN & NaN & NaN & NaN & NaN & NaN & NaN & NaN & NaN & NaN \\
TREE ASSISTANCE MICHIGAN & NaN & NaN & \$ 0.00 & \$ 0.00 & \$ 0.00 & \$ 0.00 & \$ -283.67 & \$ 0.00 & \$ 0.00 & \$ 0.00 & \$ 0.00 & \$ 0.00 & \$ 0.00 & \$ 0.00 & \$ 0.00 & \$ 0.00 & \$ 0.00 & \$ 0.00 & NaN \\
TREE ASSISTANCE PROGRAM & NaN & NaN & \$ 0.00 & \$ 0.00 & \$ 0.00 & \$ 0.00 & \$ 579.05 & \$ 485.76 & \$ -202.53 & \$ 426.94 & \$ 997.33 & \$ 803.55 & \$ 2,141.01 & \$ 100.90 & \$ 777.30 & \$ 185.31 & \$ -363.39 & \$ 10,111.11 & NaN \\
TREE ASSISTANCE PROGRAM                       & NaN & NaN & NaN & NaN & NaN & NaN & NaN & NaN & NaN & NaN & NaN & NaN & NaN & NaN & NaN & NaN & NaN & NaN & NaN \\
TREE ASSISTANCE PROGRAM - HURRICANE & NaN & NaN & \$ 0.00 & \$ 0.00 & \$ -0.00 & \$ 0.06 & \$ 0.00 & \$ 0.00 & \$ 0.00 & \$ 0.00 & \$ 0.00 & \$ 0.00 & \$ 0.00 & \$ 0.00 & \$ 0.00 & \$ 0.00 & \$ 0.00 & \$ 0.00 & NaN \\
TREE ASSISTANCE PROGRAM - MICHIGAN & NaN & NaN & \$ 0.00 & \$ 0.00 & \$ 0.00 & \$ 0.00 & \$ 0.00 & \$ 0.00 & \$ 0.00 & \$ 0.00 & \$ 0.00 & \$ 0.00 & \$ 0.00 & \$ 0.00 & \$ 0.00 & \$ 0.00 & \$ 0.00 & \$ 0.00 & NaN \\
TREE ASSISTANCE PROGRAM - ORCHARDS & NaN & NaN & \$ 0.00 & \$ 0.00 & \$ 0.00 & \$ 0.00 & \$ 0.00 & \$ 0.00 & \$ 0.00 & \$ 0.00 & \$ 0.00 & \$ 0.00 & \$ 0.00 & \$ 0.00 & \$ 0.00 & \$ 0.00 & \$ 0.00 & \$ 0.00 & NaN \\
TREE ASSISTANCE PROGRAM - PECAN & NaN & NaN & \$ 0.00 & \$ 0.00 & \$ 0.00 & \$ 0.00 & \$ 0.00 & \$ 0.00 & \$ 0.00 & \$ 0.00 & \$ 0.00 & \$ 0.00 & \$ 0.00 & \$ 0.00 & \$ 0.00 & \$ 0.00 & \$ 0.00 & \$ 0.00 & NaN \\
TREE ASSISTANCE PROGRAM - RECOVERY ACT & NaN & NaN & \$ 0.00 & \$ 0.00 & \$ 0.00 & \$ 0.00 & \$ 39,184.00 & \$ 100,000.00 & \$ 0.00 & \$ 0.00 & \$ 0.00 & \$ -953.00 & \$ 0.00 & \$ 0.00 & \$ 0.00 & \$ 0.00 & \$ 0.00 & \$ 0.00 & NaN \\
TREE ASSISTANCE PROGRAM - TIMBER & NaN & NaN & \$ 0.00 & \$ 0.00 & \$ 0.00 & \$ 0.00 & \$ 0.00 & \$ 0.00 & \$ 0.00 & \$ 0.00 & \$ 0.00 & \$ 0.00 & \$ 0.00 & \$ 0.00 & \$ 0.00 & \$ 0.00 & \$ 0.00 & \$ 0.00 & NaN \\
TREE ASSISTANCE REGULAR & NaN & NaN & \$ 0.00 & \$ 0.00 & \$ 0.00 & \$ 0.00 & \$ 0.00 & \$ 0.00 & \$ 0.00 & \$ 0.00 & \$ 0.00 & \$ 0.00 & \$ 0.00 & \$ 0.00 & \$ 0.00 & \$ 0.00 & \$ 0.00 & \$ 0.00 & NaN \\
TREE INDEMNITY PROGRAM & NaN & NaN & \$ 0.00 & \$ 0.00 & \$ 0.00 & \$ 0.00 & \$ 0.00 & \$ 0.00 & \$ 0.00 & \$ 0.00 & \$ 0.00 & \$ 0.00 & \$ 0.00 & \$ 0.00 & \$ 0.00 & \$ 0.00 & \$ 0.00 & \$ 0.00 & NaN \\
TREE THINNING INCENTIVE & NaN & NaN & NaN & NaN & NaN & NaN & NaN & NaN & NaN & NaN & NaN & NaN & NaN & NaN & NaN & NaN & NaN & NaN & NaN \\
TTPP - TOBACCO BURLEY & NaN & NaN & \$ 0.00 & \$ 0.00 & \$ 0.00 & \$ 0.00 & \$ 0.00 & \$ 0.00 & \$ 0.00 & \$ 0.00 & \$ 0.00 & \$ 0.00 & \$ 0.00 & \$ -0.24 & \$ 0.00 & \$ 0.00 & \$ 0.00 & \$ 0.00 & NaN \\
TTPP - TOBACCO FLUE CURED & NaN & NaN & \$ 0.00 & \$ 0.00 & \$ 0.00 & \$ 0.00 & \$ 0.00 & \$ 0.00 & \$ 0.00 & \$ 0.00 & \$ 0.00 & \$ 0.00 & \$ 0.00 & \$ 0.10 & \$ 0.00 & \$ 0.00 & \$ 0.00 & \$ 0.00 & NaN \\
TTPP TOBACCO PRODUCER & NaN & NaN & \$ 0.00 & \$ 0.00 & \$ 0.00 & \$ 0.00 & \$ -1,575.31 & \$ 0.00 & \$ 0.00 & \$ 0.00 & \$ 0.00 & \$ 0.00 & \$ 0.00 & \$ 0.00 & \$ 0.00 & \$ 0.00 & \$ 0.00 & \$ 0.00 & NaN \\
WEB-BASED NAP & NaN & NaN & NaN & NaN & NaN & NaN & NaN & NaN & NaN & NaN & NaN & NaN & NaN & NaN & NaN & NaN & NaN & NaN & NaN \\
WECP17OF & NaN & NaN & NaN & NaN & NaN & NaN & NaN & NaN & NaN & NaN & NaN & NaN & NaN & NaN & NaN & NaN & NaN & NaN & NaN \\
WECPCOF & NaN & NaN & NaN & NaN & NaN & NaN & NaN & NaN & NaN & NaN & NaN & NaN & NaN & NaN & NaN & NaN & NaN & NaN & NaN \\
WECPCSCOF & NaN & NaN & \$ 0.00 & \$ 0.00 & \$ 0.00 & \$ 0.00 & \$ 0.00 & \$ 0.00 & \$ 0.00 & \$ 0.00 & \$ 0.00 & \$ 0.00 & \$ 0.00 & \$ 0.00 & \$ 0.00 & \$ 0.00 & \$ -1,187.99 & \$ 14,172.70 & NaN \\
WHIP - SUGAR BEET COOPERATIVES & NaN & NaN & NaN & NaN & NaN & NaN & NaN & NaN & NaN & NaN & NaN & NaN & NaN & NaN & NaN & NaN & NaN & NaN & NaN \\
WHIP 791 & NaN & NaN & \$ 0.00 & \$ 0.00 & \$ 0.00 & \$ 0.00 & \$ 0.00 & \$ 0.00 & \$ 0.00 & \$ 0.00 & \$ 0.00 & \$ 0.00 & \$ 0.00 & \$ 0.00 & \$ 0.00 & \$ 0.00 & \$ -777.99 & \$ 99,917.33 & NaN \\
WHIP MILK LOSS & NaN & NaN & NaN & NaN & NaN & NaN & NaN & NaN & NaN & NaN & NaN & NaN & NaN & NaN & NaN & NaN & NaN & NaN & NaN \\
WHIP PLUS 3 ASSISTANCE & NaN & NaN & \$ 0.00 & \$ 0.00 & \$ 0.00 & \$ 0.00 & \$ 0.00 & \$ 0.00 & \$ 0.00 & \$ 0.00 & \$ 0.00 & \$ 0.00 & \$ 0.00 & \$ 0.00 & \$ 0.00 & \$ 0.00 & \$ 4.38 & \$ -4,433.15 & NaN \\
WILDFIRES AND HURRICANES INDEMNITY PROG & NaN & NaN & \$ 0.00 & \$ 0.00 & \$ 0.00 & \$ 0.00 & \$ 0.00 & \$ 0.00 & \$ 0.00 & \$ 0.00 & \$ 0.00 & \$ 0.00 & \$ 0.00 & \$ 0.00 & \$ 68,201.94 & \$ 936.29 & \$ 11,161.35 & \$ 73,961.39 & NaN \\
WML/WML 19 WHIP MILK LOSS & NaN & NaN & \$ 0.00 & \$ 0.00 & \$ 0.00 & \$ 0.00 & \$ 0.00 & \$ 0.00 & \$ 0.00 & \$ 0.00 & \$ 0.00 & \$ 0.00 & \$ 0.00 & \$ 0.00 & \$ 0.00 & \$ -108.42 & \$ -436.07 & \$ 0.00 & NaN \\
\end{longtable}


\section*{FOIA Data Summary}
\begin{table}[H]
\caption{Summary Stats from FOIA data}
\begin{tabular}{lrrrrr}
\toprule
 & \multicolumn{5}{r}{payment} \\
 & mean & max & sum & std & count \\
year &  &  &  &  &  \\
\midrule
2006 & \$ 788.06 & \$ 91,720.16 & \$ 441,395,789.32 & \$ 1,676.05 & 560102 \\
2007 & \$ 352.16 & \$ 25,273.56 & \$ 74,762,121.50 & \$ 864.11 & 212296 \\
2008 & \$ 867.34 & \$ 668,810.00 & \$ 10,207,878,350.00 & \$ 2,967.92 & 11769158 \\
2009 & \$ 2,022.60 & \$ 697,812.00 & \$ 9,992,060,725.00 & \$ 4,625.21 & 4940213 \\
2010 & \$ 1,606.34 & \$ 1,086,026.00 & \$ 10,615,437,806.00 & \$ 5,096.92 & 6608477 \\
2011 & \$ 1,470.96 & \$ 1,911,038.00 & \$ 7,853,899,680.98 & \$ 4,967.86 & 5339306 \\
2012 & \$ 2,178.24 & \$ 2,266,952.00 & \$ 8,055,749,016.28 & \$ 6,030.28 & 3698290 \\
2013 & \$ 2,271.05 & \$ 2,170,375.00 & \$ 7,470,720,100.00 & \$ 7,338.33 & 3289544 \\
2014 & \$ 4,208.69 & \$ 5,575,000.00 & \$ 7,179,329,103.00 & \$ 12,448.26 & 1705836 \\
2015 & \$ 3,363.29 & \$ 4,770,000.00 & \$ 8,333,282,360.00 & \$ 8,294.39 & 2477720 \\
2016 & \$ 3,071.41 & \$ 2,255,666.00 & \$ 10,711,725,040.00 & \$ 7,262.81 & 3487563 \\
2017 & \$ 2,968.00 & \$ 2,072,055.00 & \$ 9,527,872,811.00 & \$ 7,634.21 & 3210198 \\
2018 & \$ 3,400.24 & \$ 952,719.00 & \$ 11,600,600,872.59 & \$ 9,075.45 & 3411698 \\
2019 & \$ 4,878.41 & \$ 1,500,000.00 & \$ 19,979,521,019.04 & \$ 14,411.92 & 4095496 \\
2020 & \$ 4,892.06 & \$ 3,141,886.78 & \$ 37,812,702,107.61 & \$ 18,662.83 & 7729400 \\
2021 & \$ 9,509.09 & \$ 1,495,778.52 & \$ 762,857,586.43 & \$ 30,556.64 & 80224 \\
\bottomrule
\end{tabular}
\end{table}


\newpage
\section*{Public Data Summary}
\begin{table}[H]
\caption{Summary Stats from Public data}
\begin{tabular}{lrrrrr}
\toprule
 & \multicolumn{5}{r}{payment} \\
 & mean & max & sum & std & count \\
year &  &  &  &  &  \\
\midrule
2004 & \$ 1,993.27 & \$ 19,020.68 & \$ 211,287.05 & \$ 3,312.42 & 106 \\
2005 & \$ 1,858.11 & \$ 347,834.20 & \$ 1,239,418,612.72 & \$ 4,715.05 & 667033 \\
2006 & \$ 788.06 & \$ 91,720.16 & \$ 441,395,789.32 & \$ 1,676.05 & 560102 \\
2007 & \$ 352.16 & \$ 25,273.56 & \$ 74,762,121.50 & \$ 864.11 & 212296 \\
2008 & \$ 867.34 & \$ 668,810.00 & \$ 10,207,879,211.09 & \$ 2,967.92 & 11769158 \\
2009 & \$ 2,022.60 & \$ 697,812.00 & \$ 9,992,061,451.50 & \$ 4,625.21 & 4940213 \\
2010 & \$ 1,654.83 & \$ 4,925,876.00 & \$ 10,965,568,866.50 & \$ 7,522.88 & 6626408 \\
2011 & \$ 1,560.55 & \$ 5,000,000.00 & \$ 8,855,627,288.19 & \$ 7,734.06 & 5674683 \\
2012 & \$ 2,223.25 & \$ 5,000,000.00 & \$ 8,770,993,864.65 & \$ 9,345.58 & 3945131 \\
2013 & \$ 2,305.05 & \$ 4,000,000.00 & \$ 9,017,546,913.82 & \$ 9,509.81 & 3912079 \\
2014 & \$ 4,340.19 & \$ 5,575,000.00 & \$ 7,806,551,895.88 & \$ 15,419.75 & 1798664 \\
2015 & \$ 3,393.23 & \$ 4,770,000.00 & \$ 8,989,304,069.96 & \$ 10,954.09 & 2649186 \\
2016 & \$ 3,101.87 & \$ 6,000,000.00 & \$ 11,508,682,520.04 & \$ 12,510.59 & 3710243 \\
2017 & \$ 3,012.22 & \$ 3,955,922.00 & \$ 10,263,637,351.99 & \$ 10,930.82 & 3407337 \\
2018 & \$ 3,537.25 & \$ 94,000,000.00 & \$ 12,751,250,456.50 & \$ 92,089.58 & 3604845 \\
2019 & \$ 4,848.17 & \$ 3,500,000.00 & \$ 20,896,707,040.36 & \$ 15,848.31 & 4310226 \\
2020 & \$ 4,980.09 & \$ 82,324,957.00 & \$ 39,357,702,923.99 & \$ 80,209.21 & 7903003 \\
2021 & \$ 4,069.86 & \$ 9,924,038.00 & \$ 16,422,246,507.68 & \$ 14,343.61 & 4035087 \\
2022 & \$ 7,240.45 & \$ 4,243,602.50 & \$ 13,187,642,193.08 & \$ 22,089.06 & 1821385 \\
\bottomrule
\end{tabular}
\end{table}


\section*{Notes \& Takeaway}
Beginning in 2010, the two different data sources diverge. The Public Data, found at 
\href{https://www.fsa.usda.gov/news-room/efoia/electronic-reading-room/frequently-requested-information/payment-files-information/index}{this site}
has more records, a greater sum, larger mean payments, and often a higher maximum than does the 
FOIA'd data. As of July 31, 2023, I have not investigated the source of the discrepancies. My initial hunch  
is that one or more programs were omitted in the FOIA. 

Without digging deeper into the discrepancies, I recommend using the Public Data because 
\begin{itemize}
    \item We can use more recent data 
    \item The public data seems more complete 
    \item It will be easier to incorporate future data sources if USDA updates the public webpage
    \item There are no additional useful fields from the FOIA'd data that we need to make use of
\end{itemize}

\end{document}